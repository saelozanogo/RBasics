\documentclass[ignorenonframetext,]{beamer}
\setbeamertemplate{caption}[numbered]
\setbeamertemplate{caption label separator}{: }
\setbeamercolor{caption name}{fg=normal text.fg}
\beamertemplatenavigationsymbolsempty
\usepackage{lmodern}
\usepackage{amssymb,amsmath}
\usepackage{ifxetex,ifluatex}
\usepackage{fixltx2e} % provides \textsubscript
\ifnum 0\ifxetex 1\fi\ifluatex 1\fi=0 % if pdftex
  \usepackage[T1]{fontenc}
  \usepackage[utf8]{inputenc}
\else % if luatex or xelatex
  \ifxetex
    \usepackage{mathspec}
  \else
    \usepackage{fontspec}
  \fi
  \defaultfontfeatures{Ligatures=TeX,Scale=MatchLowercase}
\fi
\usetheme[]{Darmstadt}
\usefonttheme{structurebold}
% use upquote if available, for straight quotes in verbatim environments
\IfFileExists{upquote.sty}{\usepackage{upquote}}{}
% use microtype if available
\IfFileExists{microtype.sty}{%
\usepackage{microtype}
\UseMicrotypeSet[protrusion]{basicmath} % disable protrusion for tt fonts
}{}
\newif\ifbibliography
\hypersetup{
            pdftitle={Operaciones, extración y otras funcionalidades entre tipos de estructuras de datos},
            pdfauthor={Santiago Lozano},
            pdfborder={0 0 0},
            breaklinks=true}
\urlstyle{same}  % don't use monospace font for urls
\usepackage{color}
\usepackage{fancyvrb}
\newcommand{\VerbBar}{|}
\newcommand{\VERB}{\Verb[commandchars=\\\{\}]}
\DefineVerbatimEnvironment{Highlighting}{Verbatim}{commandchars=\\\{\}}
% Add ',fontsize=\small' for more characters per line
\usepackage{framed}
\definecolor{shadecolor}{RGB}{248,248,248}
\newenvironment{Shaded}{\begin{snugshade}}{\end{snugshade}}
\newcommand{\KeywordTok}[1]{\textcolor[rgb]{0.13,0.29,0.53}{\textbf{#1}}}
\newcommand{\DataTypeTok}[1]{\textcolor[rgb]{0.13,0.29,0.53}{#1}}
\newcommand{\DecValTok}[1]{\textcolor[rgb]{0.00,0.00,0.81}{#1}}
\newcommand{\BaseNTok}[1]{\textcolor[rgb]{0.00,0.00,0.81}{#1}}
\newcommand{\FloatTok}[1]{\textcolor[rgb]{0.00,0.00,0.81}{#1}}
\newcommand{\ConstantTok}[1]{\textcolor[rgb]{0.00,0.00,0.00}{#1}}
\newcommand{\CharTok}[1]{\textcolor[rgb]{0.31,0.60,0.02}{#1}}
\newcommand{\SpecialCharTok}[1]{\textcolor[rgb]{0.00,0.00,0.00}{#1}}
\newcommand{\StringTok}[1]{\textcolor[rgb]{0.31,0.60,0.02}{#1}}
\newcommand{\VerbatimStringTok}[1]{\textcolor[rgb]{0.31,0.60,0.02}{#1}}
\newcommand{\SpecialStringTok}[1]{\textcolor[rgb]{0.31,0.60,0.02}{#1}}
\newcommand{\ImportTok}[1]{#1}
\newcommand{\CommentTok}[1]{\textcolor[rgb]{0.56,0.35,0.01}{\textit{#1}}}
\newcommand{\DocumentationTok}[1]{\textcolor[rgb]{0.56,0.35,0.01}{\textbf{\textit{#1}}}}
\newcommand{\AnnotationTok}[1]{\textcolor[rgb]{0.56,0.35,0.01}{\textbf{\textit{#1}}}}
\newcommand{\CommentVarTok}[1]{\textcolor[rgb]{0.56,0.35,0.01}{\textbf{\textit{#1}}}}
\newcommand{\OtherTok}[1]{\textcolor[rgb]{0.56,0.35,0.01}{#1}}
\newcommand{\FunctionTok}[1]{\textcolor[rgb]{0.00,0.00,0.00}{#1}}
\newcommand{\VariableTok}[1]{\textcolor[rgb]{0.00,0.00,0.00}{#1}}
\newcommand{\ControlFlowTok}[1]{\textcolor[rgb]{0.13,0.29,0.53}{\textbf{#1}}}
\newcommand{\OperatorTok}[1]{\textcolor[rgb]{0.81,0.36,0.00}{\textbf{#1}}}
\newcommand{\BuiltInTok}[1]{#1}
\newcommand{\ExtensionTok}[1]{#1}
\newcommand{\PreprocessorTok}[1]{\textcolor[rgb]{0.56,0.35,0.01}{\textit{#1}}}
\newcommand{\AttributeTok}[1]{\textcolor[rgb]{0.77,0.63,0.00}{#1}}
\newcommand{\RegionMarkerTok}[1]{#1}
\newcommand{\InformationTok}[1]{\textcolor[rgb]{0.56,0.35,0.01}{\textbf{\textit{#1}}}}
\newcommand{\WarningTok}[1]{\textcolor[rgb]{0.56,0.35,0.01}{\textbf{\textit{#1}}}}
\newcommand{\AlertTok}[1]{\textcolor[rgb]{0.94,0.16,0.16}{#1}}
\newcommand{\ErrorTok}[1]{\textcolor[rgb]{0.64,0.00,0.00}{\textbf{#1}}}
\newcommand{\NormalTok}[1]{#1}

% Prevent slide breaks in the middle of a paragraph:
\widowpenalties 1 10000
\raggedbottom

\AtBeginPart{
  \let\insertpartnumber\relax
  \let\partname\relax
  \frame{\partpage}
}
\AtBeginSection{
  \ifbibliography
  \else
    \let\insertsectionnumber\relax
    \let\sectionname\relax
    \frame{\sectionpage}
  \fi
}
\AtBeginSubsection{
  \let\insertsubsectionnumber\relax
  \let\subsectionname\relax
  \frame{\subsectionpage}
}

\setlength{\parindent}{0pt}
\setlength{\parskip}{6pt plus 2pt minus 1pt}
\setlength{\emergencystretch}{3em}  % prevent overfull lines
\providecommand{\tightlist}{%
  \setlength{\itemsep}{0pt}\setlength{\parskip}{0pt}}
\setcounter{secnumdepth}{0}

\title{Operaciones, extración y otras funcionalidades entre tipos de
estructuras de datos}
\author{Santiago Lozano}
\date{21 de febrero de 2020}

\begin{document}
\frame{\titlepage}

\begin{frame}[fragile]{Función stack()}

Esta función concatena un simple vactor los valores de ciertas columnas
de un data.frame. Esta función despliega un data.frame, con los vectores
apilados en la primera columna y la segunda columna contiene un factor
que indica la columna de origen. La función unstack() realiza la
operación contraria 

\begin{Shaded}
\begin{Highlighting}[]
\NormalTok{z <-}\StringTok{ }\KeywordTok{data.frame}\NormalTok{(}\DataTypeTok{trt1=}\KeywordTok{c}\NormalTok{(}\DecValTok{1}\NormalTok{,}\DecValTok{6}\NormalTok{,}\DecValTok{3}\NormalTok{,}\DecValTok{5}\NormalTok{),}\DataTypeTok{trt2=}\KeywordTok{c}\NormalTok{(}\DecValTok{8}\NormalTok{,}\DecValTok{8}\NormalTok{,}\DecValTok{3}\NormalTok{,}\DecValTok{1}\NormalTok{))}
\NormalTok{z}
\end{Highlighting}
\end{Shaded}

\begin{verbatim}
##   trt1 trt2
## 1    1    8
## 2    6    8
## 3    3    3
## 4    5    1
\end{verbatim}

\end{frame}

\begin{frame}[fragile]{Función stack()}

\begin{Shaded}
\begin{Highlighting}[]
\KeywordTok{stack}\NormalTok{(z)}
\end{Highlighting}
\end{Shaded}

\begin{verbatim}
##   values  ind
## 1      1 trt1
## 2      6 trt1
## 3      3 trt1
## 4      5 trt1
## 5      8 trt2
## 6      8 trt2
## 7      3 trt2
## 8      1 trt2
\end{verbatim}

\end{frame}

\begin{frame}[fragile]{Función transform}

Esta función realiza transformaciones sobre las columnas de una
data.frame

\begin{Shaded}
\begin{Highlighting}[]
\NormalTok{Z <-}\StringTok{ }\KeywordTok{data.frame}\NormalTok{(}\DataTypeTok{Peso=}\KeywordTok{c}\NormalTok{(}\DecValTok{80}\NormalTok{,}\DecValTok{75}\NormalTok{,}\DecValTok{60}\NormalTok{,}\DecValTok{52}\NormalTok{),}
                \DataTypeTok{Altura=}\KeywordTok{c}\NormalTok{(}\DecValTok{180}\NormalTok{,}\DecValTok{170}\NormalTok{,}\DecValTok{165}\NormalTok{,}\DecValTok{150}\NormalTok{),}
                \DataTypeTok{Colesterol=}\KeywordTok{c}\NormalTok{(}\DecValTok{44}\NormalTok{,}\DecValTok{12}\NormalTok{,}\DecValTok{23}\NormalTok{,}\DecValTok{34}\NormalTok{),}
                \DataTypeTok{Genero=}\KeywordTok{c}\NormalTok{(}\StringTok{"M"}\NormalTok{,}\StringTok{"M"}\NormalTok{,}\StringTok{"F"}\NormalTok{,}\StringTok{"F"}\NormalTok{))}
\NormalTok{Z}
\end{Highlighting}
\end{Shaded}

\begin{verbatim}
##   Peso Altura Colesterol Genero
## 1   80    180         44      M
## 2   75    170         12      M
## 3   60    165         23      F
## 4   52    150         34      F
\end{verbatim}

\end{frame}

\begin{frame}[fragile]{Función transform}

\begin{Shaded}
\begin{Highlighting}[]
\NormalTok{Z <-}\StringTok{ }\KeywordTok{transform}\NormalTok{(Z,}\DataTypeTok{Altura=}\NormalTok{Altura}\OperatorTok{/}\DecValTok{100}\NormalTok{,}\DataTypeTok{BMI=}\NormalTok{Peso}\OperatorTok{/}\NormalTok{(Peso}\OperatorTok{/}\DecValTok{100}\NormalTok{)}\OperatorTok{^}\DecValTok{2}\NormalTok{)}
\NormalTok{Z}
\end{Highlighting}
\end{Shaded}

\begin{verbatim}
##   Peso Altura Colesterol Genero      BMI
## 1   80   1.80         44      M 125.0000
## 2   75   1.70         12      M 133.3333
## 3   60   1.65         23      F 166.6667
## 4   52   1.50         34      F 192.3077
\end{verbatim}

\end{frame}

\begin{frame}[fragile]{Función tapply}

Calcula distintas operaciónes realizadas en masas, mediante
combinaciones de variables cuantitativas con variables categóricas

\begin{Shaded}
\begin{Highlighting}[]
\NormalTok{data<-}\KeywordTok{read.table}\NormalTok{(}\StringTok{"C:/Users/santiago/Documents/}
	\StringTok{Progrmación en R/2020-I/PR04}
	\StringTok{Manipulación de datos II/temperatures.txt"}
	 \NormalTok{,}\DataTypeTok{header=}\NormalTok{T)}
\KeywordTok{attach}\NormalTok{(data)}
\KeywordTok{names}\NormalTok{(data)}
\end{Highlighting}
\end{Shaded}

\begin{verbatim}
## [1] "temperature" "lower"  "rain"  "month" "yr"
\end{verbatim}

\end{frame}

\begin{frame}[fragile]{Función tapply}

\begin{Shaded}
\begin{Highlighting}[]
\KeywordTok{tapply}\NormalTok{(temperature,month,mean)}
\end{Highlighting}
\end{Shaded}

\begin{verbatim}
##         1         2         3         4         5         
##  7.930051  8.671136 11.200508 13.813708 17.880847  
##         6         7         8         9        10
## 20.306151 22.673854 23.104924 19.344211 15.125976
##         11        12 
## 10.720702  8.299830
\end{verbatim}

\end{frame}



\begin{frame}[fragile]{Función tapply}

\begin{Shaded}
\begin{Highlighting}[]
\KeywordTok{tapply}\NormalTok{(temperature,month,min)}
\end{Highlighting}
\end{Shaded}

\begin{verbatim}
##    1    2   3   4   5    6    7    8   9   10  11  12 
## -6.8 -3.5 1.5 2.8 8.8 11.5 14.3 15.0 7.5  8.3 0.5 -1.8
\end{verbatim}


\end{frame}

\begin{frame}[fragile]{Función tapply}

\begin{Shaded}
\begin{Highlighting}[]
\KeywordTok{tapply}\NormalTok{(temperature,month,}\ControlFlowTok{function}\NormalTok{(x)}
  \KeywordTok{sqrt}\NormalTok{(}\KeywordTok{var}\NormalTok{(x)}\OperatorTok{/}\KeywordTok{length}\NormalTok{(x)))}
\end{Highlighting}
\end{Shaded}

\begin{verbatim}
##         1         2         3         4         5         
## 0.1401489 0.1414445 0.1358934 0.1476242 0.1673197  
##         6         7         8         9        10
## 0.1596439 0.1539661 0.1516091 0.1309294 0.1155612
##         11        12 
## 0.1291703 0.1398438

\end{verbatim}

\end{frame}




\begin{frame}[fragile]{Función tapply}

mediante esta función podemos producir tablas multidimensionales
simplemente reemplzando una variable categórica por una lista de
variables categóricas

\begin{Shaded}
\begin{Highlighting}[]
\KeywordTok{tapply}\NormalTok{(temperature,}\KeywordTok{list}\NormalTok{(yr,month),mean)[,}\DecValTok{1}\OperatorTok{:}\DecValTok{6}\NormalTok{]}
\end{Highlighting}
\end{Shaded}

\end{frame}

\begin{frame}[fragile]{Función tapply}

\begin{verbatim}
          1      2      3     4     5     6
1987  3.170  6.871  8.132 14.92 15.60 17.73
1988  8.048  8.248  9.959 12.74 17.31 18.71
1989  8.841  9.482 11.919 11.03 20.43 21.23
1990  9.445 11.021 12.487 13.80 20.19 18.57
1991  6.980  4.817 12.022 13.13 15.58 16.88
1992  6.964  8.686 11.477 13.30 20.45 22.21
1993 10.115  6.984 11.207 14.10 17.75 21.10
1994  8.825  7.217 11.806 12.67 16.23 20.86
1995  8.309 10.436 10.662 14.77 18.73 19.93
1996  7.019  6.065  8.4870 13.96 14.31 21.96
1997  4.932 10.171 13.378 15.07 18.19 19.90
1998  8.759 11.247 11.715 12.53 19.46 19.30
1999  9.523  8.485 11.790 14.60 18.94 20.00
2000  8.229 10.328 11.900 12.50 18.21 20.63
2001  7.067  9.121  9.012912.66 18.96 20.52
2002  9.067 11.399 12.315 15.67 16.80 19.67
2003  8.012  8.171 13.425 15.60 17.36 22.80
2004  8.261  8.993 10.354 15.10 17.98 21.73
2005  9.116  7.032 10.787 13.73 17.12 22.00
\end{verbatim}

\end{frame}


\begin{frame}[fragile]{Función tapply}

para corregir el inconveniente de trabajar con los missing data usamos
el argumento na.rm=TRUE

\begin{Shaded}
\begin{Highlighting}[]
\KeywordTok{tapply}\NormalTok{(temperature,yr,mean,}\DataTypeTok{na.rm=}\OtherTok{TRUE}\NormalTok{)}
\end{Highlighting}
\end{Shaded}

\begin{verbatim}
##     1987     1988     1989     1990     1991     1992     
## 13.27014 13.79126 15.54986 15.62986 14.11945 14.61612 
##     1993     1994     1995     1996     1997     1998     
## 14.30984 15.12877 15.81260 13.98082 15.63918 15.02568 
##     1999     2000     2001     2002    2003     2004      
## 15.63736 14.94071 14.90849 15.47589 16.03260 15.25109 
##     2005  
## 15.06000 

\end{verbatim}

\end{frame}

\begin{frame}{Función tapply}

Usted puede querer eliminar ciertos valores extremos antes de calcular
la media (pues la media aritmética es bastante sensible a datos
atípicos), para ello, el argumento trim especifica la fracción de los
datos (entre 0 y 0.5 que usted quiere omitir). Los valores extremos son
omitidos en prioridad

\end{frame}

\begin{frame}[fragile]{Función tapply}

\begin{Shaded}
\begin{Highlighting}[]
\KeywordTok{tapply}\NormalTok{(temperature,yr,mean,}\DataTypeTok{trim=}\FloatTok{0.2}\NormalTok{)}
\end{Highlighting}
\end{Shaded}

\begin{verbatim}
##     1987     1988     1989     1990     1991     1992     
## 13.45068 13.74500 14.99726 15.16301 13.92237 14.32091
##     1993     1994     1995     1996     1997     1998     
## 14.28000 14.64658 15.25571 13.75845 15.54064 14.91500 
##     1999     2000     2001     2002    2003     2004      
## 15.44364 14.59318 14.63333 15.33927 15.70959 15.04136 
##     2005  
## 15.02009 
\end{verbatim}

\end{frame}

\begin{frame}[fragile]{Función aggregate()}

La función aggregate() genera un data.frame en sub poblaciones de
acuerdo a un factor (especificado por el argumento by) y aplica una
función dada para cada subpoblación

\begin{Shaded}
\begin{Highlighting}[]
\NormalTok{Z <-}\StringTok{ }\KeywordTok{data.frame}\NormalTok{(}\DataTypeTok{Peso=}\KeywordTok{c}\NormalTok{(}\DecValTok{80}\NormalTok{,}\DecValTok{75}\NormalTok{,}\DecValTok{60}\NormalTok{,}\DecValTok{52}\NormalTok{),}
                \DataTypeTok{Altura=}\KeywordTok{c}\NormalTok{(}\DecValTok{180}\NormalTok{,}\DecValTok{170}\NormalTok{,}\DecValTok{165}\NormalTok{,}\DecValTok{150}\NormalTok{),}
                \DataTypeTok{Colesterol=}\KeywordTok{c}\NormalTok{(}\DecValTok{44}\NormalTok{,}\DecValTok{12}\NormalTok{,}\DecValTok{23}\NormalTok{,}\DecValTok{34}\NormalTok{),}
                \DataTypeTok{Genero=}\KeywordTok{c}\NormalTok{(}\StringTok{"M"}\NormalTok{,}\StringTok{"M"}\NormalTok{,}\StringTok{"F"}\NormalTok{,}\StringTok{"F"}\NormalTok{))}
\NormalTok{Z}
\end{Highlighting}
\end{Shaded}

\begin{verbatim}
##   Peso Altura Colesterol Genero
## 1   80    180         44      M
## 2   75    170         12      M
## 3   60    165         23      F
## 4   52    150         34      F
\end{verbatim}

\end{frame}

\begin{frame}[fragile]{Función aggregate}

\begin{Shaded}
\begin{Highlighting}[]
\KeywordTok{aggregate}\NormalTok{(Z[,}\OperatorTok{-}\DecValTok{4}\NormalTok{],}\DataTypeTok{by=}\KeywordTok{list}\NormalTok{(}\DataTypeTok{Gender=}\NormalTok{Z[,}\DecValTok{4}\NormalTok{]),}\DataTypeTok{FUN=}\NormalTok{mean)}
\end{Highlighting}
\end{Shaded}

\begin{verbatim}
##   Gender Peso Altura Colesterol
## 1      F 56.0  157.5       28.5
## 2      M 77.5  175.0       28.0
\end{verbatim}

\end{frame}

\begin{frame}{Función aggregate()}

Suponga que tenemos dos variables respuesta (y y z) y dos variables
explicativas (x y w) que pueden ser usadas para realizar un resumen
estadístico. la función aggregate() funciona de distintas maneras

\begin{itemize}
\tightlist
\item
  one to one: aggregate(y \textasciitilde{} x, mean)
\item
  one to many: aggregate(y \textasciitilde{} x + w, mean)
\item
  many to one: aggregate(cbind(y,z) \textasciitilde{} x, mean)
\item
  many to many: aggregate(cbind(y,z) \textasciitilde{} x + w, mean)
\end{itemize}

\end{frame}

\begin{frame}[fragile]{Función aggregate()}

\begin{Shaded}
\begin{Highlighting}[]
\NormalTok{data2<-}\KeywordTok{read.table}\NormalTok{(}\StringTok{"C:/Users/santiago/Documents/}
	\StringTok{"Progrmación en R/2020-I/PR04}
	\StringTok{Manipulación de datos II/pHDaphnia.txt"}\NormalTok{,}\DataTypeTok{header=}\NormalTok{T)}
\KeywordTok{names}\NormalTok{(data2)}
\end{Highlighting}
\end{Shaded}

\begin{verbatim}
## [1] "Growth.rate" "Water" "Detergent" "Daphnia"  "pH"
\end{verbatim}

\begin{Shaded}
\begin{Highlighting}[]
\KeywordTok{aggregate}\NormalTok{(Growth.rate}\OperatorTok{~}\NormalTok{Water,data2,mean)}
\end{Highlighting}
\end{Shaded}

\begin{verbatim}
##   Water Growth.rate
## 1  Tyne    3.685862
## 2  Wear    4.017948
\end{verbatim}

\end{frame}

\begin{frame}[fragile]{Función aggregate()}

\begin{Shaded}
\begin{Highlighting}[]
\KeywordTok{aggregate}\NormalTok{(Growth.rate}\OperatorTok{~}\NormalTok{Water}\OperatorTok{+}\NormalTok{Detergent,data2,mean)}
\end{Highlighting}
\end{Shaded}

\begin{verbatim}
##   Water Detergent Growth.rate
## 1  Tyne    BrandA    3.661807
## 2  Wear    BrandA    4.107857
## 3  Tyne    BrandB    3.911116
## 4  Wear    BrandB    4.108972
## 5  Tyne    BrandC    3.814321
## 6  Wear    BrandC    4.094704
## 7  Tyne    BrandD    3.356203
## 8  Wear    BrandD    3.760259
\end{verbatim}

\end{frame}

\begin{frame}[fragile]{Función aggregate()}

\begin{Shaded}
\begin{Highlighting}[]
\KeywordTok{aggregate}\NormalTok{(}\KeywordTok{cbind}\NormalTok{(pH,Growth.rate)}\OperatorTok{~}\NormalTok{Water}\OperatorTok{+}\NormalTok{Detergent,}
\NormalTok{          data2,mean)}
\end{Highlighting}
\end{Shaded}

\begin{verbatim}
##   Water Detergent       pH Growth.rate
## 1  Tyne    BrandA 4.883908    3.661807
## 2  Wear    BrandA 5.054835    4.107857
## 3  Tyne    BrandB 5.043797    3.911116
## 4  Wear    BrandB 4.892346    4.108972
## 5  Tyne    BrandC 4.847069    3.814321
## 6  Wear    BrandC 4.912128    4.094704
## 7  Tyne    BrandD 4.809144    3.356203
## 8  Wear    BrandD 5.097039    3.760259
\end{verbatim}

\end{frame}

\end{document}
