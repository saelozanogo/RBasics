\documentclass[]{article}
\usepackage{lmodern}
\usepackage{amssymb,amsmath}
\usepackage{ifxetex,ifluatex}
\usepackage{fixltx2e} % provides \textsubscript
\ifnum 0\ifxetex 1\fi\ifluatex 1\fi=0 % if pdftex
  \usepackage[T1]{fontenc}
  \usepackage[utf8]{inputenc}
\else % if luatex or xelatex
  \ifxetex
    \usepackage{mathspec}
  \else
    \usepackage{fontspec}
  \fi
  \defaultfontfeatures{Ligatures=TeX,Scale=MatchLowercase}
\fi
% use upquote if available, for straight quotes in verbatim environments
\IfFileExists{upquote.sty}{\usepackage{upquote}}{}
% use microtype if available
\IfFileExists{microtype.sty}{%
\usepackage{microtype}
\UseMicrotypeSet[protrusion]{basicmath} % disable protrusion for tt fonts
}{}
\usepackage[margin=1in]{geometry}
\usepackage{hyperref}
\hypersetup{unicode=true,
            pdftitle={Parcial},
            pdfauthor={Santiago Lozano-Fulanito González},
            pdfborder={0 0 0},
            breaklinks=true}
\urlstyle{same}  % don't use monospace font for urls
\usepackage{color}
\usepackage{fancyvrb}
\newcommand{\VerbBar}{|}
\newcommand{\VERB}{\Verb[commandchars=\\\{\}]}
\DefineVerbatimEnvironment{Highlighting}{Verbatim}{commandchars=\\\{\}}
% Add ',fontsize=\small' for more characters per line
\usepackage{framed}
\definecolor{shadecolor}{RGB}{248,248,248}
\newenvironment{Shaded}{\begin{snugshade}}{\end{snugshade}}
\newcommand{\KeywordTok}[1]{\textcolor[rgb]{0.13,0.29,0.53}{\textbf{#1}}}
\newcommand{\DataTypeTok}[1]{\textcolor[rgb]{0.13,0.29,0.53}{#1}}
\newcommand{\DecValTok}[1]{\textcolor[rgb]{0.00,0.00,0.81}{#1}}
\newcommand{\BaseNTok}[1]{\textcolor[rgb]{0.00,0.00,0.81}{#1}}
\newcommand{\FloatTok}[1]{\textcolor[rgb]{0.00,0.00,0.81}{#1}}
\newcommand{\ConstantTok}[1]{\textcolor[rgb]{0.00,0.00,0.00}{#1}}
\newcommand{\CharTok}[1]{\textcolor[rgb]{0.31,0.60,0.02}{#1}}
\newcommand{\SpecialCharTok}[1]{\textcolor[rgb]{0.00,0.00,0.00}{#1}}
\newcommand{\StringTok}[1]{\textcolor[rgb]{0.31,0.60,0.02}{#1}}
\newcommand{\VerbatimStringTok}[1]{\textcolor[rgb]{0.31,0.60,0.02}{#1}}
\newcommand{\SpecialStringTok}[1]{\textcolor[rgb]{0.31,0.60,0.02}{#1}}
\newcommand{\ImportTok}[1]{#1}
\newcommand{\CommentTok}[1]{\textcolor[rgb]{0.56,0.35,0.01}{\textit{#1}}}
\newcommand{\DocumentationTok}[1]{\textcolor[rgb]{0.56,0.35,0.01}{\textbf{\textit{#1}}}}
\newcommand{\AnnotationTok}[1]{\textcolor[rgb]{0.56,0.35,0.01}{\textbf{\textit{#1}}}}
\newcommand{\CommentVarTok}[1]{\textcolor[rgb]{0.56,0.35,0.01}{\textbf{\textit{#1}}}}
\newcommand{\OtherTok}[1]{\textcolor[rgb]{0.56,0.35,0.01}{#1}}
\newcommand{\FunctionTok}[1]{\textcolor[rgb]{0.00,0.00,0.00}{#1}}
\newcommand{\VariableTok}[1]{\textcolor[rgb]{0.00,0.00,0.00}{#1}}
\newcommand{\ControlFlowTok}[1]{\textcolor[rgb]{0.13,0.29,0.53}{\textbf{#1}}}
\newcommand{\OperatorTok}[1]{\textcolor[rgb]{0.81,0.36,0.00}{\textbf{#1}}}
\newcommand{\BuiltInTok}[1]{#1}
\newcommand{\ExtensionTok}[1]{#1}
\newcommand{\PreprocessorTok}[1]{\textcolor[rgb]{0.56,0.35,0.01}{\textit{#1}}}
\newcommand{\AttributeTok}[1]{\textcolor[rgb]{0.77,0.63,0.00}{#1}}
\newcommand{\RegionMarkerTok}[1]{#1}
\newcommand{\InformationTok}[1]{\textcolor[rgb]{0.56,0.35,0.01}{\textbf{\textit{#1}}}}
\newcommand{\WarningTok}[1]{\textcolor[rgb]{0.56,0.35,0.01}{\textbf{\textit{#1}}}}
\newcommand{\AlertTok}[1]{\textcolor[rgb]{0.94,0.16,0.16}{#1}}
\newcommand{\ErrorTok}[1]{\textcolor[rgb]{0.64,0.00,0.00}{\textbf{#1}}}
\newcommand{\NormalTok}[1]{#1}
\usepackage{graphicx,grffile}
\makeatletter
\def\maxwidth{\ifdim\Gin@nat@width>\linewidth\linewidth\else\Gin@nat@width\fi}
\def\maxheight{\ifdim\Gin@nat@height>\textheight\textheight\else\Gin@nat@height\fi}
\makeatother
% Scale images if necessary, so that they will not overflow the page
% margins by default, and it is still possible to overwrite the defaults
% using explicit options in \includegraphics[width, height, ...]{}
\setkeys{Gin}{width=\maxwidth,height=\maxheight,keepaspectratio}
\IfFileExists{parskip.sty}{%
\usepackage{parskip}
}{% else
\setlength{\parindent}{0pt}
\setlength{\parskip}{6pt plus 2pt minus 1pt}
}
\setlength{\emergencystretch}{3em}  % prevent overfull lines
\providecommand{\tightlist}{%
  \setlength{\itemsep}{0pt}\setlength{\parskip}{0pt}}
\setcounter{secnumdepth}{0}
% Redefines (sub)paragraphs to behave more like sections
\ifx\paragraph\undefined\else
\let\oldparagraph\paragraph
\renewcommand{\paragraph}[1]{\oldparagraph{#1}\mbox{}}
\fi
\ifx\subparagraph\undefined\else
\let\oldsubparagraph\subparagraph
\renewcommand{\subparagraph}[1]{\oldsubparagraph{#1}\mbox{}}
\fi

%%% Use protect on footnotes to avoid problems with footnotes in titles
\let\rmarkdownfootnote\footnote%
\def\footnote{\protect\rmarkdownfootnote}

%%% Change title format to be more compact
\usepackage{titling}

% Create subtitle command for use in maketitle
\providecommand{\subtitle}[1]{
  \posttitle{
    \begin{center}\large#1\end{center}
    }
}

\setlength{\droptitle}{-2em}

  \title{Parcial}
    \pretitle{\vspace{\droptitle}\centering\huge}
  \posttitle{\par}
    \author{Santiago Lozano-Fulanito González}
    \preauthor{\centering\large\emph}
  \postauthor{\par}
      \predate{\centering\large\emph}
  \postdate{\par}
    \date{21 de marzo de 2020}


\begin{document}
\maketitle

Este examen parcial se debe realizar INDIVIDUAL parciales
iguales tendrán nota de cero, y lo deben entregar antes de las 9:00 PM,
deben entregar un script, es decir un .R, con las codificaciónes
respectivas y debidamente comentado, es obligatorio usar las funciones
vistas en las diapositiva

\subsection{1}\label{section}

\begin{enumerate}
\def\labelenumi{\alph{enumi}.}
\tightlist
\item
  Cree 5 vectore. Uno de caracteres, uno de numeric, uno de integer y
  uno de complex, cada uno con más de 5 elementos
\end{enumerate}

\begin{Shaded}
\begin{Highlighting}[]
\NormalTok{x<-}\StringTok{ }\KeywordTok{c}\NormalTok{(}\FloatTok{1.4}\NormalTok{,}\FloatTok{3.8}\NormalTok{,}\FloatTok{4.7}\NormalTok{,}\FloatTok{5.9}\NormalTok{,}\FloatTok{1.6}\NormalTok{,}\FloatTok{3.6}\NormalTok{)}
\NormalTok{y<-}\KeywordTok{c}\NormalTok{(}\StringTok{"Santiago"}\NormalTok{,}\StringTok{"Enrique"}\NormalTok{,}\StringTok{"Lozano"}\NormalTok{, }\StringTok{"González"}\NormalTok{,}\StringTok{"Fulanito"}\NormalTok{)}
\NormalTok{z<-}\StringTok{ }\KeywordTok{as.integer}\NormalTok{(}\DecValTok{1}\OperatorTok{:}\DecValTok{6}\NormalTok{)}
\NormalTok{w<-}\StringTok{ }\KeywordTok{c}\NormalTok{(}\DecValTok{1}\OperatorTok{+}\NormalTok{5i,}\DecValTok{2}\OperatorTok{+}\NormalTok{6i,}\DecValTok{3}\OperatorTok{+}\NormalTok{7i,}\DecValTok{4}\OperatorTok{+}\NormalTok{8i,}\DecValTok{5}\OperatorTok{+}\NormalTok{1i,}\DecValTok{6}\OperatorTok{+}\NormalTok{2i,}\DecValTok{7}\OperatorTok{+}\NormalTok{3i)}
\end{Highlighting}
\end{Shaded}

\begin{enumerate}
\def\labelenumi{\alph{enumi}.}
\setcounter{enumi}{1}
\tightlist
\item
  Chequee si los vectores son del tipo especificado ¿Cómo lo hace en R?
\end{enumerate}

\begin{Shaded}
\begin{Highlighting}[]
\KeywordTok{class}\NormalTok{(x)}
\end{Highlighting}
\end{Shaded}

\begin{verbatim}
## [1] "numeric"
\end{verbatim}

\begin{Shaded}
\begin{Highlighting}[]
\KeywordTok{class}\NormalTok{(y)}
\end{Highlighting}
\end{Shaded}

\begin{verbatim}
## [1] "character"
\end{verbatim}

\begin{Shaded}
\begin{Highlighting}[]
\KeywordTok{class}\NormalTok{(z)}
\end{Highlighting}
\end{Shaded}

\begin{verbatim}
## [1] "integer"
\end{verbatim}

\begin{Shaded}
\begin{Highlighting}[]
\KeywordTok{class}\NormalTok{(w)}
\end{Highlighting}
\end{Shaded}

\begin{verbatim}
## [1] "complex"
\end{verbatim}

\begin{enumerate}
\def\labelenumi{\alph{enumi}.}
\setcounter{enumi}{2}
\tightlist
\item
  Cree una lista de varios tipos de estructuras de datos, mínimo 4
  elementos debe tener la lista
\end{enumerate}

\begin{Shaded}
\begin{Highlighting}[]
\NormalTok{Mat<-}\KeywordTok{matrix}\NormalTok{(}\DecValTok{1}\OperatorTok{:}\DecValTok{12}\NormalTok{,}\DataTypeTok{nrow=}\DecValTok{4}\NormalTok{,}\DataTypeTok{byrow=}\OtherTok{TRUE}\NormalTok{)}
\NormalTok{L<-}\KeywordTok{list}\NormalTok{(}\DecValTok{12}\NormalTok{,}\KeywordTok{c}\NormalTok{(}\DecValTok{34}\NormalTok{,}\DecValTok{67}\NormalTok{),Mat,}\DecValTok{1}\OperatorTok{:}\DecValTok{15}\NormalTok{,}\KeywordTok{list}\NormalTok{(}\DecValTok{10}\NormalTok{,}\DecValTok{11}\NormalTok{))}
\NormalTok{L}
\end{Highlighting}
\end{Shaded}

\begin{verbatim}
## [[1]]
## [1] 12
## 
## [[2]]
## [1] 34 67
## 
## [[3]]
##      [,1] [,2] [,3]
## [1,]    1    2    3
## [2,]    4    5    6
## [3,]    7    8    9
## [4,]   10   11   12
## 
## [[4]]
##  [1]  1  2  3  4  5  6  7  8  9 10 11 12 13 14 15
## 
## [[5]]
## [[5]][[1]]
## [1] 10
## 
## [[5]][[2]]
## [1] 11
\end{verbatim}

\begin{enumerate}
\def\labelenumi{\alph{enumi}.}
\setcounter{enumi}{3}
\tightlist
\item
  Cree una matriz de 3 filas por 4 columna con datos arreglados del 1 al
  12 en forma horizontal
\end{enumerate}

\begin{Shaded}
\begin{Highlighting}[]
\NormalTok{M <-}\StringTok{ }\KeywordTok{matrix}\NormalTok{(}\DecValTok{1}\OperatorTok{:}\DecValTok{12}\NormalTok{,}\DataTypeTok{nrow =} \DecValTok{3}\NormalTok{,}\DataTypeTok{byrow =}\NormalTok{ T)}
\NormalTok{M}
\end{Highlighting}
\end{Shaded}

\begin{verbatim}
##      [,1] [,2] [,3] [,4]
## [1,]    1    2    3    4
## [2,]    5    6    7    8
## [3,]    9   10   11   12
\end{verbatim}

\begin{enumerate}
\def\labelenumi{\alph{enumi}.}
\setcounter{enumi}{4}
\tightlist
\item
  Cree una matriz de 3 filas y 4 columnas con números arbitrarios y con
  acomodación vertical
\end{enumerate}

\begin{Shaded}
\begin{Highlighting}[]
\NormalTok{x<-}\KeywordTok{runif}\NormalTok{(}\DecValTok{12}\NormalTok{)}
\NormalTok{Ma <-}\StringTok{ }\KeywordTok{matrix}\NormalTok{(x,}\DataTypeTok{nrow =} \DecValTok{3}\NormalTok{,}\DataTypeTok{byrow =}\NormalTok{ F)}
\NormalTok{Ma}
\end{Highlighting}
\end{Shaded}

\begin{verbatim}
##           [,1]       [,2]       [,3]        [,4]
## [1,] 0.8231838 0.80720703 0.36101877 0.240627309
## [2,] 0.5406383 0.07906838 0.08840597 0.805871282
## [3,] 0.5175967 0.95005285 0.76252430 0.001787599
\end{verbatim}

\begin{enumerate}
\def\labelenumi{\alph{enumi}.}
\setcounter{enumi}{5}
\tightlist
\item
  Cree un vector de factores con calificaciones de estudiantes
  (Deficiente, insuficiente,aceptable, sobresaliente, excelente), que
  contenga mínimos 13 entradas distribuidas de manera aleatoria
\end{enumerate}

\begin{Shaded}
\begin{Highlighting}[]
\NormalTok{calificaciones <-}\StringTok{ }\KeywordTok{factor}\NormalTok{(}\KeywordTok{c}\NormalTok{(}\StringTok{"deficiente"}\NormalTok{,}\StringTok{"excelente"}\NormalTok{,}\StringTok{"insuficiente"}\NormalTok{,}\StringTok{"deficiente"}\NormalTok{,}\StringTok{"aceptable"}\\
\NormalTok{,}\StringTok{"suficiente"}\NormalTok{,}\StringTok{"excelente"}\NormalTok{,}\StringTok{"insuficiente"}\NormalTok{,}\StringTok{"excelente"}\NormalTok{,}\StringTok{"insuficiente"}\NormalTok{,}\StringTok{"deficiente"}\NormalTok{,}\StringTok{"insuficiente"}\NormalTok{,}\\
\StringTok{"insuficiente"}\NormalTok{,}\StringTok{"excelente"}\NormalTok{,}\StringTok{"excelente"}\NormalTok{))}
\NormalTok{calificaciones}
\end{Highlighting}
\end{Shaded}

\begin{verbatim}
##  [1] deficiente   excelente    insuficiente deficiente   aceptable   
##  [6] suficiente   excelente    insuficiente excelente    insuficiente
## [11] deficiente   insuficiente insuficiente excelente    excelente   
## Levels: aceptable deficiente excelente insuficiente suficiente
\end{verbatim}

\begin{enumerate}
\def\labelenumi{\alph{enumi}.}
\setcounter{enumi}{6}
\tightlist
\item
  Cree un vertor de tipo ordered con niveles de riesgo en un hospital
  (bajo, medio, alto) donde bajo es el menor, medio el que le sigue y
  alto es mayor, debe tener mínimos 8 entradas
\end{enumerate}

\begin{Shaded}
\begin{Highlighting}[]
\NormalTok{z <-}\StringTok{ }\KeywordTok{ordered}\NormalTok{(}\KeywordTok{c}\NormalTok{(}\StringTok{"alto"}\NormalTok{,}\StringTok{"bajo"}\NormalTok{,}\StringTok{"bajo"}\NormalTok{,}\StringTok{"medio"}\NormalTok{,}\StringTok{"alto"}\NormalTok{,}\StringTok{"bajo"}\NormalTok{,}\StringTok{"medio"}\NormalTok{,}\StringTok{"bajo"}\NormalTok{,}\StringTok{"medio"}\NormalTok{,}\StringTok{"bajo"}\NormalTok{,}\\
\StringTok{"alto"}\NormalTok{,}\StringTok{"bajo"}\NormalTok{,}\StringTok{"medio"}\NormalTok{,}\StringTok{"alto"}\NormalTok{,}\StringTok{"bajo"}\NormalTok{,}\StringTok{"medio"}\NormalTok{,}\StringTok{"alto"}\NormalTok{),}
\DataTypeTok{levels=}\KeywordTok{c}\NormalTok{(}\StringTok{"bajo"}\NormalTok{,}\StringTok{"medio"}\NormalTok{,}\StringTok{"alto"}\NormalTok{))}
\NormalTok{z}
\end{Highlighting}
\end{Shaded}

\begin{verbatim}
##  [1] alto  bajo  bajo  medio alto  bajo  medio bajo  medio bajo  alto 
## [12] bajo  medio alto  bajo  medio alto 
## Levels: bajo < medio < alto
\end{verbatim}

\subsection{2}\label{section-1}

Cree la siguiente base de datos

\begin{Shaded}
\begin{Highlighting}[]
\NormalTok{base11 <-}\StringTok{ }\KeywordTok{read.table}\NormalTok{(}\StringTok{"base11.txt"}\NormalTok{,}\DataTypeTok{header =}\NormalTok{ T,}\DataTypeTok{sep =} \StringTok{" "}\NormalTok{)}
\NormalTok{base11}
\end{Highlighting}
\end{Shaded}

\begin{verbatim}
##        id   sexo año.de.nacimiento fecha.de.confirmación
## 539   539 female              1975            2020-02-23
## 1190 1190 female              1960            2020-02-26
## 457   457 female              1963            2020-02-23
## 230   230 female              1961            2020-02-22
## 117   117 female              1980            2020-02-21
## 487   487 female              1967            2020-02-23
## 217   217 female              1962            2020-02-22
## 532   532   male              1956            2020-02-23
\end{verbatim}

\begin{Shaded}
\begin{Highlighting}[]
\KeywordTok{attach}\NormalTok{(base11)}
\NormalTok{id <-}\StringTok{ }\KeywordTok{c}\NormalTok{(}\DecValTok{539}\NormalTok{,}\DecValTok{1190}\NormalTok{,}\DecValTok{457}\NormalTok{,}\DecValTok{230}\NormalTok{,}\DecValTok{117}\NormalTok{,}\DecValTok{487}\NormalTok{,}\DecValTok{217}\NormalTok{,}\DecValTok{532}\NormalTok{)}
\NormalTok{sexo <-}\KeywordTok{c}\NormalTok{(}\StringTok{"female"}\NormalTok{,}\StringTok{"female"}\NormalTok{,}\StringTok{"female"}\NormalTok{,}\StringTok{"female"}\NormalTok{,}\StringTok{"female"}\NormalTok{,}\StringTok{"female"}\NormalTok{,}\StringTok{"female"}\NormalTok{,}\StringTok{"male"}\NormalTok{)}
\NormalTok{ano.de.nacimiento <-}\StringTok{ }\KeywordTok{c}\NormalTok{(}\DecValTok{1975}\NormalTok{,}\DecValTok{1960}\NormalTok{,}\DecValTok{1963}\NormalTok{,}\DecValTok{1961}\NormalTok{,}\DecValTok{1980}\NormalTok{,}\DecValTok{1967}\NormalTok{,}\DecValTok{1962}\NormalTok{,}\DecValTok{1956}\NormalTok{)}
\NormalTok{fecha.de.confirmacion <-}\StringTok{ }\KeywordTok{c}\NormalTok{(}\StringTok{"2020-02-23"}\NormalTok{,}\StringTok{"2020-02-26"}\NormalTok{,}\StringTok{"2020-02-23"}\NormalTok{,}\StringTok{"2020-02-22"}\NormalTok{, }\StringTok{"2020-02-21"}\NormalTok{,}\\
\StringTok{"2020-02-23"}\NormalTok{,}\StringTok{"2020-02-22"}\NormalTok{,}\StringTok{"2020-02-23"}\NormalTok{)}
\NormalTok{data <-}\StringTok{ }\KeywordTok{data.frame}\NormalTok{(}\StringTok{"id"}\NormalTok{=id,}\StringTok{"sexo"}\NormalTok{=sexo,}\StringTok{"año.de.nacimiento"}\NormalTok{=ano.de.nacimiento,}\\

\StringTok{"fecha.de.confirmación"=fecha.de.confirmacion)}
\StringTok{data}
\end{Highlighting}
\end{Shaded}

\begin{verbatim}
##     id   sexo año.de.nacimiento fecha.de.confirmación
## 1  539 female              1975            2020-02-23
## 2 1190 female              1960            2020-02-26
## 3  457 female              1963            2020-02-23
## 4  230 female              1961            2020-02-22
## 5  117 female              1980            2020-02-21
## 6  487 female              1967            2020-02-23
## 7  217 female              1962            2020-02-22
## 8  532   male              1956            2020-02-23
\end{verbatim}

\subsection{3}\label{section-2}

Importe la base de datos base22

\begin{Shaded}
\begin{Highlighting}[]
\NormalTok{base22 <-}\StringTok{ }\KeywordTok{read.table}\NormalTok{(}\StringTok{"base22.txt"}\NormalTok{,}\DataTypeTok{header =}\NormalTok{ T,}\DataTypeTok{sep =} \StringTok{" "}\NormalTok{)}
\NormalTok{base22}
\end{Highlighting}
\end{Shaded}

\begin{verbatim}
##       país   sexo   Estado   id
## 531  Korea female isolated  531
## 457  Korea female isolated  457
## 481  Korea female isolated  481
## 117  Korea female isolated  117
## 1184 Korea   male isolated 1184
## 539  Korea female isolated  539
## 224  Korea female isolated  224
## 217  Korea female isolated  217
\end{verbatim}

\begin{enumerate}
\def\labelenumi{\alph{enumi}.}
\tightlist
\item
  Extraiga los id de las mujeres (sexo==``female'')
\end{enumerate}

\begin{Shaded}
\begin{Highlighting}[]
\KeywordTok{attach}\NormalTok{(base22)}
\end{Highlighting}
\end{Shaded}

\begin{verbatim}
## The following objects are masked _by_ .GlobalEnv:
## 
##     id, sexo
\end{verbatim}

\begin{verbatim}
## The following objects are masked from base11:
## 
##     id, sexo
\end{verbatim}

\begin{Shaded}
\begin{Highlighting}[]
\NormalTok{id[sexo}\OperatorTok{==}\StringTok{"female"}\NormalTok{]}
\end{Highlighting}
\end{Shaded}

\begin{verbatim}
## [1]  539 1190  457  230  117  487  217
\end{verbatim}

\begin{enumerate}
\def\labelenumi{\alph{enumi}.}
\setcounter{enumi}{1}
\tightlist
\item
  Extraiga las filas de la 4 a 7 y las columnas de 2 a 4
\end{enumerate}

\begin{Shaded}
\begin{Highlighting}[]
\NormalTok{data[}\DecValTok{4}\OperatorTok{:}\DecValTok{7}\NormalTok{,}\DecValTok{2}\OperatorTok{:}\DecValTok{4}\NormalTok{]}
\end{Highlighting}
\end{Shaded}

\begin{verbatim}
##     sexo año.de.nacimiento fecha.de.confirmación
## 4 female              1961            2020-02-22
## 5 female              1980            2020-02-21
## 6 female              1967            2020-02-23
## 7 female              1962            2020-02-22
\end{verbatim}

\subsection{4}\label{section-3}

De la matriz

\begin{Shaded}
\begin{Highlighting}[]
\NormalTok{ymatrix <-}\StringTok{ }\KeywordTok{matrix}\NormalTok{(}\DataTypeTok{data =} \KeywordTok{c}\NormalTok{(}\DecValTok{6}\NormalTok{,}\DecValTok{34}\NormalTok{,}\DecValTok{923}\NormalTok{,}\DecValTok{5}\NormalTok{,}\DecValTok{0}\NormalTok{, }\DecValTok{112}\OperatorTok{:}\DecValTok{116}\NormalTok{, }\DecValTok{5}\NormalTok{,}\DecValTok{9}\NormalTok{,}\DecValTok{34}\NormalTok{,}\DecValTok{76}\NormalTok{,}\DecValTok{2}\NormalTok{, }\DecValTok{545}\OperatorTok{:}\DecValTok{549}\NormalTok{),}\DataTypeTok{nrow=}\DecValTok{5}\NormalTok{)}
\NormalTok{ymatrix}
\end{Highlighting}
\end{Shaded}

\begin{verbatim}
##      [,1] [,2] [,3] [,4]
## [1,]    6  112    5  545
## [2,]   34  113    9  546
## [3,]  923  114   34  547
## [4,]    5  115   76  548
## [5,]    0  116    2  549
\end{verbatim}

\begin{enumerate}
\def\labelenumi{\alph{enumi}.}
\tightlist
\item
  Saque el promedio de cada fila
\end{enumerate}

\begin{Shaded}
\begin{Highlighting}[]
\KeywordTok{apply}\NormalTok{(ymatrix,}\DataTypeTok{MARGIN =} \DecValTok{1}\NormalTok{,}\DataTypeTok{FUN =}\NormalTok{ mean)}
\end{Highlighting}
\end{Shaded}

\begin{verbatim}
## [1] 167.00 175.50 404.50 186.00 166.75
\end{verbatim}

\begin{enumerate}
\def\labelenumi{\alph{enumi}.}
\setcounter{enumi}{1}
\tightlist
\item
  ordene cada fila de mayor a menor
\end{enumerate}

\begin{Shaded}
\begin{Highlighting}[]
\KeywordTok{apply}\NormalTok{(ymatrix,}\DataTypeTok{MARGIN =} \DecValTok{1}\NormalTok{,}\DataTypeTok{FUN =}\ControlFlowTok{function}\NormalTok{(x)\{}\KeywordTok{rev}\NormalTok{(}\KeywordTok{sort}\NormalTok{(x))\})}
\end{Highlighting}
\end{Shaded}

\begin{verbatim}
##      [,1] [,2] [,3] [,4] [,5]
## [1,]  545  546  923  548  549
## [2,]  112  113  547  115  116
## [3,]    6   34  114   76    2
## [4,]    5    9   34    5    0
\end{verbatim}

\subsection{5.}\label{section-4}

Importe la base de datos worms

\begin{Shaded}
\begin{Highlighting}[]
\NormalTok{worms <-}\StringTok{ }\KeywordTok{read.table}\NormalTok{(}\StringTok{"worms.txt"}\NormalTok{,}\DataTypeTok{header =}\NormalTok{ T,}\DataTypeTok{dec =} \StringTok{"."}\NormalTok{)}
\KeywordTok{attach}\NormalTok{(worms)}
\NormalTok{worms}
\end{Highlighting}
\end{Shaded}

\begin{verbatim}
##           Field.Name Area Slope Vegetation Soil.pH  Damp Worm.density
## 1        Nashs.Field  3.6    11  Grassland     4.1 FALSE            4
## 2     Silwood.Bottom  5.1     2     Arable     5.2 FALSE            7
## 3      Nursery.Field  2.8     3  Grassland     4.3 FALSE            2
## 4        Rush.Meadow  2.4     5     Meadow     4.9  TRUE            5
## 5    Gunness.Thicket  3.8     0      Scrub     4.2 FALSE            6
## 6           Oak.Mead  3.1     2  Grassland     3.9 FALSE            2
## 7       Church.Field  3.5     3  Grassland     4.2 FALSE            3
## 8            Ashurst  2.1     0     Arable     4.8 FALSE            4
## 9        The.Orchard  1.9     0    Orchard     5.7 FALSE            9
## 10     Rookery.Slope  1.5     4  Grassland     5.0  TRUE            7
## 11       Garden.Wood  2.9    10      Scrub     5.2 FALSE            8
## 12      North.Gravel  3.3     1  Grassland     4.1 FALSE            1
## 13      South.Gravel  3.7     2  Grassland     4.0 FALSE            2
## 14 Observatory.Ridge  1.8     6  Grassland     3.8 FALSE            0
## 15        Pond.Field  4.1     0     Meadow     5.0  TRUE            6
## 16      Water.Meadow  3.9     0     Meadow     4.9  TRUE            8
## 17         Cheapside  2.2     8      Scrub     4.7  TRUE            4
## 18        Pound.Hill  4.4     2     Arable     4.5 FALSE            5
## 19        Gravel.Pit  2.9     1  Grassland     3.5 FALSE            1
## 20         Farm.Wood  0.8    10      Scrub     5.1  TRUE            3
\end{verbatim}

\begin{enumerate}
\def\labelenumi{\alph{enumi}.}
\tightlist
\item
  saque el promedio de los elementos de Area con respecto a Vegetation,
  es decir cuál es el promedio de las Areas de Grassland y así
  sucesivamente
\end{enumerate}

\begin{Shaded}
\begin{Highlighting}[]
\KeywordTok{tapply}\NormalTok{(Area,Vegetation,mean)}
\end{Highlighting}
\end{Shaded}

\begin{verbatim}
##    Arable Grassland    Meadow   Orchard     Scrub 
##  3.866667  2.911111  3.466667  1.900000  2.425000
\end{verbatim}


\end{document}
