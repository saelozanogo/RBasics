\documentclass[]{article}
\usepackage{lmodern}
\usepackage{amssymb,amsmath}
\usepackage{ifxetex,ifluatex}
\usepackage{fixltx2e} % provides \textsubscript
\ifnum 0\ifxetex 1\fi\ifluatex 1\fi=0 % if pdftex
  \usepackage[T1]{fontenc}
  \usepackage[utf8]{inputenc}
\else % if luatex or xelatex
  \ifxetex
    \usepackage{mathspec}
  \else
    \usepackage{fontspec}
  \fi
  \defaultfontfeatures{Ligatures=TeX,Scale=MatchLowercase}
\fi
% use upquote if available, for straight quotes in verbatim environments
\IfFileExists{upquote.sty}{\usepackage{upquote}}{}
% use microtype if available
\IfFileExists{microtype.sty}{%
\usepackage{microtype}
\UseMicrotypeSet[protrusion]{basicmath} % disable protrusion for tt fonts
}{}
\usepackage[margin=1in]{geometry}
\usepackage{hyperref}
\hypersetup{unicode=true,
            pdftitle={Parcial primer corte},
            pdfauthor={Santiago Lozano},
            pdfborder={0 0 0},
            breaklinks=true}
\urlstyle{same}  % don't use monospace font for urls
\usepackage{color}
\usepackage{fancyvrb}
\newcommand{\VerbBar}{|}
\newcommand{\VERB}{\Verb[commandchars=\\\{\}]}
\DefineVerbatimEnvironment{Highlighting}{Verbatim}{commandchars=\\\{\}}
% Add ',fontsize=\small' for more characters per line
\usepackage{framed}
\definecolor{shadecolor}{RGB}{248,248,248}
\newenvironment{Shaded}{\begin{snugshade}}{\end{snugshade}}
\newcommand{\KeywordTok}[1]{\textcolor[rgb]{0.13,0.29,0.53}{\textbf{#1}}}
\newcommand{\DataTypeTok}[1]{\textcolor[rgb]{0.13,0.29,0.53}{#1}}
\newcommand{\DecValTok}[1]{\textcolor[rgb]{0.00,0.00,0.81}{#1}}
\newcommand{\BaseNTok}[1]{\textcolor[rgb]{0.00,0.00,0.81}{#1}}
\newcommand{\FloatTok}[1]{\textcolor[rgb]{0.00,0.00,0.81}{#1}}
\newcommand{\ConstantTok}[1]{\textcolor[rgb]{0.00,0.00,0.00}{#1}}
\newcommand{\CharTok}[1]{\textcolor[rgb]{0.31,0.60,0.02}{#1}}
\newcommand{\SpecialCharTok}[1]{\textcolor[rgb]{0.00,0.00,0.00}{#1}}
\newcommand{\StringTok}[1]{\textcolor[rgb]{0.31,0.60,0.02}{#1}}
\newcommand{\VerbatimStringTok}[1]{\textcolor[rgb]{0.31,0.60,0.02}{#1}}
\newcommand{\SpecialStringTok}[1]{\textcolor[rgb]{0.31,0.60,0.02}{#1}}
\newcommand{\ImportTok}[1]{#1}
\newcommand{\CommentTok}[1]{\textcolor[rgb]{0.56,0.35,0.01}{\textit{#1}}}
\newcommand{\DocumentationTok}[1]{\textcolor[rgb]{0.56,0.35,0.01}{\textbf{\textit{#1}}}}
\newcommand{\AnnotationTok}[1]{\textcolor[rgb]{0.56,0.35,0.01}{\textbf{\textit{#1}}}}
\newcommand{\CommentVarTok}[1]{\textcolor[rgb]{0.56,0.35,0.01}{\textbf{\textit{#1}}}}
\newcommand{\OtherTok}[1]{\textcolor[rgb]{0.56,0.35,0.01}{#1}}
\newcommand{\FunctionTok}[1]{\textcolor[rgb]{0.00,0.00,0.00}{#1}}
\newcommand{\VariableTok}[1]{\textcolor[rgb]{0.00,0.00,0.00}{#1}}
\newcommand{\ControlFlowTok}[1]{\textcolor[rgb]{0.13,0.29,0.53}{\textbf{#1}}}
\newcommand{\OperatorTok}[1]{\textcolor[rgb]{0.81,0.36,0.00}{\textbf{#1}}}
\newcommand{\BuiltInTok}[1]{#1}
\newcommand{\ExtensionTok}[1]{#1}
\newcommand{\PreprocessorTok}[1]{\textcolor[rgb]{0.56,0.35,0.01}{\textit{#1}}}
\newcommand{\AttributeTok}[1]{\textcolor[rgb]{0.77,0.63,0.00}{#1}}
\newcommand{\RegionMarkerTok}[1]{#1}
\newcommand{\InformationTok}[1]{\textcolor[rgb]{0.56,0.35,0.01}{\textbf{\textit{#1}}}}
\newcommand{\WarningTok}[1]{\textcolor[rgb]{0.56,0.35,0.01}{\textbf{\textit{#1}}}}
\newcommand{\AlertTok}[1]{\textcolor[rgb]{0.94,0.16,0.16}{#1}}
\newcommand{\ErrorTok}[1]{\textcolor[rgb]{0.64,0.00,0.00}{\textbf{#1}}}
\newcommand{\NormalTok}[1]{#1}
\usepackage{graphicx,grffile}
\makeatletter
\def\maxwidth{\ifdim\Gin@nat@width>\linewidth\linewidth\else\Gin@nat@width\fi}
\def\maxheight{\ifdim\Gin@nat@height>\textheight\textheight\else\Gin@nat@height\fi}
\makeatother
% Scale images if necessary, so that they will not overflow the page
% margins by default, and it is still possible to overwrite the defaults
% using explicit options in \includegraphics[width, height, ...]{}
\setkeys{Gin}{width=\maxwidth,height=\maxheight,keepaspectratio}
\IfFileExists{parskip.sty}{%
\usepackage{parskip}
}{% else
\setlength{\parindent}{0pt}
\setlength{\parskip}{6pt plus 2pt minus 1pt}
}
\setlength{\emergencystretch}{3em}  % prevent overfull lines
\providecommand{\tightlist}{%
  \setlength{\itemsep}{0pt}\setlength{\parskip}{0pt}}
\setcounter{secnumdepth}{0}
% Redefines (sub)paragraphs to behave more like sections
\ifx\paragraph\undefined\else
\let\oldparagraph\paragraph
\renewcommand{\paragraph}[1]{\oldparagraph{#1}\mbox{}}
\fi
\ifx\subparagraph\undefined\else
\let\oldsubparagraph\subparagraph
\renewcommand{\subparagraph}[1]{\oldsubparagraph{#1}\mbox{}}
\fi

%%% Use protect on footnotes to avoid problems with footnotes in titles
\let\rmarkdownfootnote\footnote%
\def\footnote{\protect\rmarkdownfootnote}

%%% Change title format to be more compact
\usepackage{titling}

% Create subtitle command for use in maketitle
\providecommand{\subtitle}[1]{
  \posttitle{
    \begin{center}\large#1\end{center}
    }
}

\setlength{\droptitle}{-2em}

  \title{Parcial primer corte}
    \pretitle{\vspace{\droptitle}\centering\huge}
  \posttitle{\par}
    \author{Santiago Lozano}
    \preauthor{\centering\large\emph}
  \postauthor{\par}
      \predate{\centering\large\emph}
  \postdate{\par}
    \date{14 de marzo de 2020}


\begin{document}
\maketitle

\begin{Shaded}
\begin{Highlighting}[]
\NormalTok{base11 <-}\StringTok{ }\KeywordTok{read.table}\NormalTok{(}\StringTok{"base11.txt"}\NormalTok{,}\DataTypeTok{header =}\NormalTok{ T,}\DataTypeTok{sep =} \StringTok{" "}\NormalTok{)}
\NormalTok{base11}
\end{Highlighting}
\end{Shaded}

\begin{verbatim}
##        id   sexo año.de.nacimiento fecha.de.confirmación
## 539   539 female              1975            2020-02-23
## 1190 1190 female              1960            2020-02-26
## 457   457 female              1963            2020-02-23
## 230   230 female              1961            2020-02-22
## 117   117 female              1980            2020-02-21
## 487   487 female              1967            2020-02-23
## 217   217 female              1962            2020-02-22
## 532   532   male              1956            2020-02-23
\end{verbatim}

\begin{Shaded}
\begin{Highlighting}[]
\NormalTok{base22 <-}\StringTok{ }\KeywordTok{read.table}\NormalTok{(}\StringTok{"base22.txt"}\NormalTok{,}\DataTypeTok{header =}\NormalTok{ T,}\DataTypeTok{sep =} \StringTok{" "}\NormalTok{)}
\NormalTok{base22}
\end{Highlighting}
\end{Shaded}

\begin{verbatim}
##       país   sexo   Estado   id
## 531  Korea female isolated  531
## 457  Korea female isolated  457
## 481  Korea female isolated  481
## 117  Korea female isolated  117
## 1184 Korea   male isolated 1184
## 539  Korea female isolated  539
## 224  Korea female isolated  224
## 217  Korea female isolated  217
\end{verbatim}

\subsection{1}\label{section}

Escriba falso o verdadero según corresponda

\begin{enumerate}
\def\labelenumi{\alph{enumi}.}
\tightlist
\item
  si tengo el vector \texttt{x\ \textless{}-\ c(5,7,3,1,4,3,8)} y a este
  le aplico \texttt{x{[}-c(1,2){]}}, obtengo el último y el penúltimo
  vector respectivamente (Falso)
\end{enumerate}

\begin{Shaded}
\begin{Highlighting}[]
\NormalTok{x <-}\StringTok{ }\KeywordTok{c}\NormalTok{(}\DecValTok{5}\NormalTok{,}\DecValTok{7}\NormalTok{,}\DecValTok{3}\NormalTok{,}\DecValTok{1}\NormalTok{,}\DecValTok{4}\NormalTok{,}\DecValTok{3}\NormalTok{,}\DecValTok{8}\NormalTok{)}
\NormalTok{x[}\OperatorTok{-}\KeywordTok{c}\NormalTok{(}\DecValTok{1}\NormalTok{,}\DecValTok{2}\NormalTok{)]}
\end{Highlighting}
\end{Shaded}

\begin{verbatim}
## [1] 3 1 4 3 8
\end{verbatim}

\begin{enumerate}
\def\labelenumi{\alph{enumi}.}
\setcounter{enumi}{1}
\tightlist
\item
  Teniendo la siguiente lista
  \texttt{L\textless{}-list(12,c(34,67),Mat,1:15,list(10,11))}, donde\\
  \texttt{Mat\textless{}-matrix(1:12,nrow=4,byrow=TRUE)}, si aplico
  \texttt{L{[}{[}4{]}{]}{[}7:10{]}} obtengo los números del 7 al 10
  (verdadero)
\end{enumerate}

\begin{Shaded}
\begin{Highlighting}[]
\NormalTok{Mat<-}\KeywordTok{matrix}\NormalTok{(}\DecValTok{1}\OperatorTok{:}\DecValTok{12}\NormalTok{,}\DataTypeTok{nrow=}\DecValTok{4}\NormalTok{,}\DataTypeTok{byrow=}\OtherTok{TRUE}\NormalTok{)}
\NormalTok{L<-}\KeywordTok{list}\NormalTok{(}\DecValTok{12}\NormalTok{,}\KeywordTok{c}\NormalTok{(}\DecValTok{34}\NormalTok{,}\DecValTok{67}\NormalTok{),Mat,}\DecValTok{1}\OperatorTok{:}\DecValTok{15}\NormalTok{,}\KeywordTok{list}\NormalTok{(}\DecValTok{10}\NormalTok{,}\DecValTok{11}\NormalTok{))}
\NormalTok{L[[}\DecValTok{4}\NormalTok{]][}\DecValTok{7}\OperatorTok{:}\DecValTok{10}\NormalTok{]}
\end{Highlighting}
\end{Shaded}

\begin{verbatim}
## [1]  7  8  9 10
\end{verbatim}

\begin{enumerate}
\def\labelenumi{\alph{enumi}.}
\setcounter{enumi}{2}
\tightlist
\item
  si en
  \texttt{ymatrix\ =\ matrix(data\ =\ c(6,34,923,5,0,\ 112:116,\ 5,9,34,76,2,\ 545:549),\ nrow\ =\ 5)}
  aplico \texttt{ymatrix{[}c(1,5),c(1,3){]}} obtengo la matriz (Falso)
\end{enumerate}

\begin{Shaded}
\begin{Highlighting}[]
\NormalTok{ymatrix =}\StringTok{ }\KeywordTok{matrix}\NormalTok{(}\DataTypeTok{data =} \KeywordTok{c}\NormalTok{(}\DecValTok{6}\NormalTok{,}\DecValTok{34}\NormalTok{,}\DecValTok{923}\NormalTok{,}\DecValTok{5}\NormalTok{,}\DecValTok{0}\NormalTok{, }\DecValTok{112}\OperatorTok{:}\DecValTok{116}\NormalTok{, }\DecValTok{5}\NormalTok{,}\DecValTok{9}\NormalTok{,}\DecValTok{34}\NormalTok{,}\DecValTok{76}\NormalTok{,}\DecValTok{2}\NormalTok{, }\DecValTok{545}\OperatorTok{:}\DecValTok{549}\NormalTok{), }\DataTypeTok{nrow =} \DecValTok{5}\NormalTok{)}
\NormalTok{ymatrix[}\KeywordTok{c}\NormalTok{(}\DecValTok{1}\NormalTok{,}\DecValTok{5}\NormalTok{),}\KeywordTok{c}\NormalTok{(}\DecValTok{1}\NormalTok{,}\DecValTok{3}\NormalTok{)]}
\end{Highlighting}
\end{Shaded}

\begin{verbatim}
##      [,1] [,2]
## [1,]    6    5
## [2,]    0    2
\end{verbatim}

\begin{verbatim}
##      [,1] [,2]
## [1,]  923  114
## [2,]    5  115
\end{verbatim}

\begin{enumerate}
\def\labelenumi{\alph{enumi}.}
\setcounter{enumi}{3}
\tightlist
\item
  los dataframes son una estructura de datos (Verdadero)
\end{enumerate}

\subsection{2}\label{section-1}

¿Cuál es el output de las siguientes líneas de código

\begin{enumerate}
\def\labelenumi{\alph{enumi}.}
\item
\end{enumerate}

\begin{Shaded}
\begin{Highlighting}[]
\NormalTok{vecA <-}\StringTok{ }\KeywordTok{c}\NormalTok{(}\DecValTok{1}\NormalTok{,}\DecValTok{3}\NormalTok{,}\DecValTok{6}\NormalTok{,}\DecValTok{2}\NormalTok{,}\DecValTok{7}\NormalTok{,}\DecValTok{4}\NormalTok{,}\DecValTok{8}\NormalTok{,}\DecValTok{1}\NormalTok{,}\DecValTok{0}\NormalTok{)}
\NormalTok{vecB <-}\StringTok{ }\KeywordTok{c}\NormalTok{(vecA, }\DecValTok{4}\NormalTok{, }\DecValTok{1}\NormalTok{)}
\NormalTok{vecC <-}\StringTok{ }\KeywordTok{c}\NormalTok{(vecA[}\DecValTok{1}\OperatorTok{:}\DecValTok{4}\NormalTok{], }\DecValTok{8}\NormalTok{, }\DecValTok{5}\NormalTok{, vecA[}\DecValTok{5}\OperatorTok{:}\DecValTok{9}\NormalTok{])}
\NormalTok{vecC[vecB}\OperatorTok{>}\DecValTok{4}\NormalTok{]}
\end{Highlighting}
\end{Shaded}

\begin{verbatim}
## [1] 6 8 7
\end{verbatim}

\begin{enumerate}
\def\labelenumi{\alph{enumi}.}
\setcounter{enumi}{1}
\item
\end{enumerate}

\begin{Shaded}
\begin{Highlighting}[]
\NormalTok{a <-}\StringTok{ }\KeywordTok{c}\NormalTok{()}
\NormalTok{a <-}\StringTok{ }\KeywordTok{c}\NormalTok{(a,}\DecValTok{2}\NormalTok{)}
\NormalTok{a <-}\StringTok{ }\KeywordTok{c}\NormalTok{(a,}\DecValTok{7}\NormalTok{)}
\NormalTok{a}
\end{Highlighting}
\end{Shaded}

\begin{verbatim}
## [1] 2 7
\end{verbatim}

\begin{enumerate}
\def\labelenumi{\alph{enumi}.}
\setcounter{enumi}{2}
\item
\end{enumerate}

\begin{Shaded}
\begin{Highlighting}[]
\NormalTok{y1 <-}\StringTok{ }\KeywordTok{c}\NormalTok{(}\DecValTok{1}\NormalTok{,}\DecValTok{2}\NormalTok{,}\DecValTok{3}\NormalTok{,}\OtherTok{NA}\NormalTok{)}
\NormalTok{y2 <-}\StringTok{ }\KeywordTok{c}\NormalTok{(}\DecValTok{5}\NormalTok{,}\DecValTok{6}\NormalTok{,}\OtherTok{NA}\NormalTok{,}\DecValTok{8}\NormalTok{)}
\NormalTok{y3 <-}\StringTok{ }\KeywordTok{c}\NormalTok{(}\DecValTok{9}\NormalTok{,}\OtherTok{NA}\NormalTok{,}\DecValTok{11}\NormalTok{,}\DecValTok{12}\NormalTok{)}
\NormalTok{y4 <-}\StringTok{ }\KeywordTok{c}\NormalTok{(}\OtherTok{NA}\NormalTok{,}\DecValTok{14}\NormalTok{,}\DecValTok{15}\NormalTok{,}\DecValTok{16}\NormalTok{)}
\NormalTok{full.frame <-}\StringTok{ }\KeywordTok{data.frame}\NormalTok{(y1,y2,y3,y4)}
\NormalTok{reduced.frame <-}\StringTok{ }\NormalTok{full.frame[}\OperatorTok{!}\KeywordTok{is.na}\NormalTok{(full.frame}\OperatorTok{$}\NormalTok{y1),]}
\NormalTok{reduced.frame}
\end{Highlighting}
\end{Shaded}

\begin{verbatim}
##   y1 y2 y3 y4
## 1  1  5  9 NA
## 2  2  6 NA 14
## 3  3 NA 11 15
\end{verbatim}

\begin{enumerate}
\def\labelenumi{\alph{enumi}.}
\setcounter{enumi}{3}
\tightlist
\item
  de
\end{enumerate}

\begin{Shaded}
\begin{Highlighting}[]
\NormalTok{base11}
\end{Highlighting}
\end{Shaded}

\begin{verbatim}
##        id   sexo año.de.nacimiento fecha.de.confirmación
## 539   539 female              1975            2020-02-23
## 1190 1190 female              1960            2020-02-26
## 457   457 female              1963            2020-02-23
## 230   230 female              1961            2020-02-22
## 117   117 female              1980            2020-02-21
## 487   487 female              1967            2020-02-23
## 217   217 female              1962            2020-02-22
## 532   532   male              1956            2020-02-23
\end{verbatim}

que obtengo si ejecuto

\begin{Shaded}
\begin{Highlighting}[]
\NormalTok{base11}\OperatorTok{$}\NormalTok{id[base11}\OperatorTok{$}\NormalTok{sexo}\OperatorTok{==}\StringTok{"female"}\NormalTok{]}
\end{Highlighting}
\end{Shaded}

\begin{verbatim}
## [1]  539 1190  457  230  117  487  217
\end{verbatim}

Sugerencia: Recuerde que si escribo \texttt{base11\$id} estoy accediendo
a la variable id del dataframe base11

\begin{enumerate}
\def\labelenumi{\alph{enumi}.}
\setcounter{enumi}{4}
\tightlist
\item
  nombre 3 tipos de datos
\end{enumerate}

\subsection{3}\label{section-2}

Tome los dos siguientes dataframes

\begin{Shaded}
\begin{Highlighting}[]
\NormalTok{base11}
\end{Highlighting}
\end{Shaded}

\begin{verbatim}
##        id   sexo año.de.nacimiento fecha.de.confirmación
## 539   539 female              1975            2020-02-23
## 1190 1190 female              1960            2020-02-26
## 457   457 female              1963            2020-02-23
## 230   230 female              1961            2020-02-22
## 117   117 female              1980            2020-02-21
## 487   487 female              1967            2020-02-23
## 217   217 female              1962            2020-02-22
## 532   532   male              1956            2020-02-23
\end{verbatim}

\begin{Shaded}
\begin{Highlighting}[]
\NormalTok{base22}
\end{Highlighting}
\end{Shaded}

\begin{verbatim}
##       país   sexo   Estado   id
## 531  Korea female isolated  531
## 457  Korea female isolated  457
## 481  Korea female isolated  481
## 117  Korea female isolated  117
## 1184 Korea   male isolated 1184
## 539  Korea female isolated  539
## 224  Korea female isolated  224
## 217  Korea female isolated  217
\end{verbatim}

3.1. Realice un merge (combinación) entre los dos dataframes de manera
ordinaria

\begin{Shaded}
\begin{Highlighting}[]
\KeywordTok{merge}\NormalTok{(base11,base22)}
\end{Highlighting}
\end{Shaded}

\begin{verbatim}
##    id   sexo año.de.nacimiento fecha.de.confirmación  país   Estado
## 1 117 female              1980            2020-02-21 Korea isolated
## 2 217 female              1962            2020-02-22 Korea isolated
## 3 457 female              1963            2020-02-23 Korea isolated
## 4 539 female              1975            2020-02-23 Korea isolated
\end{verbatim}

3.2. En el caso anterior por defecto R identifica las columnas comunes
por el nombre común de la variable, recordemos que podemos forzar las
variables comunes con el argumento \textbf{by}, haga un merge forzando a
id como única columna común y despliegue el posible resultado

\begin{Shaded}
\begin{Highlighting}[]
\KeywordTok{merge}\NormalTok{(base11,base22,}\DataTypeTok{by=}\KeywordTok{c}\NormalTok{(}\StringTok{"id"}\NormalTok{))}
\end{Highlighting}
\end{Shaded}

\begin{verbatim}
##    id sexo.x año.de.nacimiento fecha.de.confirmación  país sexo.y   Estado
## 1 117 female              1980            2020-02-21 Korea female isolated
## 2 217 female              1962            2020-02-22 Korea female isolated
## 3 457 female              1963            2020-02-23 Korea female isolated
## 4 539 female              1975            2020-02-23 Korea female isolated
\end{verbatim}

\subsection{4}\label{section-3}

La función apply(), aplica una función dada (con el argumento FUN) a
todas la filas (MARGIN=1) o todas las columnas (MARGIN=2), vamos a
querer aplicar una operación masiva a una matriz usando apply()

2.1. Escriba la codificación en R para obtenener la siguiente matriz

\begin{verbatim}
##      [,1] [,2] [,3] [,4]
## [1,]    6  112    5  545
## [2,]   34  113    9  546
## [3,]  923  114   34  547
## [4,]    5  115   76  548
## [5,]    0  116    2  549
\end{verbatim}

2.2. Escriba la codificación en R para obtener la media de cada fila

\begin{Shaded}
\begin{Highlighting}[]
\KeywordTok{apply}\NormalTok{(mymatrix, }\DataTypeTok{MARGIN =} \DecValTok{1}\NormalTok{,}\DataTypeTok{FUN =}\NormalTok{ mean)}
\end{Highlighting}
\end{Shaded}

\begin{verbatim}
## [1] 167.00 175.50 404.50 186.00 166.75
\end{verbatim}

2.3. Escriba la codificación en R para obtener la media de cada columna

\begin{Shaded}
\begin{Highlighting}[]
\KeywordTok{apply}\NormalTok{(mymatrix, }\DataTypeTok{MARGIN =} \DecValTok{2}\NormalTok{,}\DataTypeTok{FUN =}\NormalTok{ mean)}
\end{Highlighting}
\end{Shaded}

\begin{verbatim}
## [1] 193.6 114.0  25.2 547.0
\end{verbatim}

2.4. la función sort nos permite ordenar los elemento de un vector, por
ejemplo

\begin{Shaded}
\begin{Highlighting}[]
\KeywordTok{sort}\NormalTok{(}\KeywordTok{c}\NormalTok{(}\DecValTok{7}\NormalTok{,}\DecValTok{2}\NormalTok{,}\DecValTok{6}\NormalTok{,}\DecValTok{0}\NormalTok{,}\DecValTok{4}\NormalTok{,}\DecValTok{8}\NormalTok{,}\DecValTok{5}\NormalTok{,}\DecValTok{9}\NormalTok{))}
\end{Highlighting}
\end{Shaded}

\begin{verbatim}
## [1] 0 2 4 5 6 7 8 9
\end{verbatim}

De esta manera escriba la codificación para ordenar los elementos de
cada columna y como quedarían

\begin{Shaded}
\begin{Highlighting}[]
\KeywordTok{apply}\NormalTok{(mymatrix, }\DataTypeTok{MARGIN =} \DecValTok{2}\NormalTok{,}\DataTypeTok{FUN =}\NormalTok{ sort)}
\end{Highlighting}
\end{Shaded}

\begin{verbatim}
##      [,1] [,2] [,3] [,4]
## [1,]    0  112    2  545
## [2,]    5  113    5  546
## [3,]    6  114    9  547
## [4,]   34  115   34  548
## [5,]  923  116   76  549
\end{verbatim}


\end{document}
