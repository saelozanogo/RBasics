% !TEX TS-program = pdflatex
% !TEX encoding = UTF-8 Unicode

% This is a simple template for a LaTeX document using the "article" class.
% See "book", "report", "letter" for other types of document.

\documentclass[12pt]{article} % use larger type; default would be 10pt
%\usepackage{tgtermes} % times font
\usepackage[utf8]{inputenc} % set input encoding (not needed with XeLaTeX)

%%% Examples of Article customizations
% These packages are optional, depending whether you want the features they provide.
% See the LaTeX Companion or other references for full information.

%%% PAGE DIMENSIONS
\usepackage{geometry} % to change the page dimensions
\geometry{a4paper} % or letterpaper (US) or a5paper or....
% \geometry{margin=2in} % for example, change the margins to 2 inches all round
% \geometry{landscape} % set up the page for landscape
%   read geometry.pdf for detailed page layout information

\usepackage{graphicx} % support the \includegraphics command and options

% \usepackage[parfill]{parskip} % Activate to begin paragraphs with an empty line rather than an indent
\usepackage[spanish]{babel}
%%% PACKAGES
\usepackage{amsmath,amssymb}
\usepackage{booktabs} % for much better looking tables
\usepackage{array} % for better arrays (eg matrices) in maths
\usepackage{paralist} % very flexible & customisable lists (eg. enumerate/itemize, etc.)
\usepackage{verbatim} % adds environment for commenting out blocks of text & for better verbatim
\usepackage{subfig} % make it possible to include more than one captioned figure/table in a single float
\usepackage{listings}
% These packages are all incorporated in the memoir class to one degree or another...

%%% HEADERS & FOOTERS
\usepackage{fancyhdr} % This should be set AFTER setting up the page geometry
\pagestyle{fancy} % options: empty , plain , fancy
\renewcommand{\headrulewidth}{0pt} % customise the layout...
\lhead{}\chead{}\rhead{}
\lfoot{}\cfoot{\thepage}\rfoot{}

%%% SECTION TITLE APPEARANCE
\usepackage{sectsty}
\allsectionsfont{\sffamily\mdseries\upshape} % (See the fntguide.pdf for font help)
% (This matches ConTeXt defaults)

%%% ToC (table of contents) APPEARANCE
\usepackage[nottoc,notlof,notlot]{tocbibind} % Put the bibliography in the ToC
\usepackage[titles,subfigure]{tocloft} % Alter the style of the Table of Contents
\renewcommand{\cftsecfont}{\rmfamily\mdseries\upshape}
\renewcommand{\cftsecpagefont}{\rmfamily\mdseries\upshape} % No bold!

%%% END Article customizations

%%% The "real" document content comes below...

\title{Taller primer corte}
\author{Santiago Enrique Lozano González \\
	Programación en R \\
	Universidad Piloto de Colombia Seccional Alto Magdalena  \\
	santiago-lozano@unipiloto.edu.co \\
	}
\date{\today}
\begin{document}
\maketitle
\begin{enumerate}
\item (Piensa como R) Resuelva estos ejercicios sin tipearlos en R, si afirma que hay algún error, explique por qué sucede este
\begin{enumerate}
\item ¿Cuál es la salida de la instrucción: \textbf{1:3\^{}2}?
\item ¿Cuál es la salida de la instrucción: \textbf{(1:3)*2}?
\item ¿Cuál es la salida de la instrucción: \textbf{var$<$-3}? ¿y posteriormente de \textbf{Var*2}?
\item ¿Cuál es la salida de la instrucción: \textbf{x$<$-2}? \\
¿y posteriormente de \textbf{2x$<$-2*x}?
\item ¿Cuál es la salida de la instrucción: \textbf{root.of.four $<$- sqrt(4)}? ¿y posteriormente de \textbf{root.of.four}?
\item ¿Cuál es la salida de la instrucción: \textbf{x$<$-1}? ¿y posteriormente de \textbf{x$<$ -1}?
\item ¿Cuál es la salida de la instrucción: \textbf{Número par $<$- 16}?
\item ¿Cuál es la salida de la instrucción: \textbf{"Número par" $<$- 16}?
\item ¿Cuál es la salida de la instrucción: \textbf{"2x" $<$- 14}?
\item ¿Cuál es la salida de la instrucción: \textbf{Número par}?
\item ¿Cuál es el resultado de la siguiente operación $(6+4)^2+(11+10/2)$?
\item ¿Cuál es el resultado de la siguiente operación $11*10-12/3$?
\item ¿Cuál es el resultado de la siguiente operación $4*(12*6-4^2)+9$?
\end{enumerate}
\item Nombre los tipos de datos en R
\item Dé la instrucción en R la cual de la siguiente salida
\item Nombre las estructuras de datos disponibles en R
\begin{center}
\begin{figure}
\includegraphics[scale=0.44]{matriz.png}
\end{figure}
\end{center}
\item desamos analizar las caraterísticas de una muestra de niños. Estos niños fueron enviados a una exminación médica en su primer año de jardín en 1996-1997 en escuela de la ciudad de Bogotá. La muestra contiene información de 10 niños entre 3 y 4 años\par
La siguinete información está disponible para cada niño
\begin{itemize}
\item Género M mujer, H 
\end{itemize}
\end{enumerate}
\end{document}
