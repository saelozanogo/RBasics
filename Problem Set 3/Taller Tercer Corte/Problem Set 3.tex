% !TEX TS-program = pdflatex
% !TEX encoding = UTF-8 Unicode

% This is a simple template for a LaTeX document using the "article" class.
% See "book", "report", "letter" for other types of document.

\documentclass[11pt]{article} % use larger type; default would be 10pt

\usepackage[utf8]{inputenc} % set input encoding (not needed with XeLaTeX)

%%% Examples of Article customizations
% These packages are optional, depending whether you want the features they provide.
% See the LaTeX Companion or other references for full information.

%%% PAGE DIMENSIONS
\usepackage{geometry} % to change the page dimensions
\geometry{a4paper} % or letterpaper (US) or a5paper or....
% \geometry{margin=2in} % for example, change the margins to 2 inches all round
% \geometry{landscape} % set up the page for landscape
%   read geometry.pdf for detailed page layout information

\usepackage{graphicx} % support the \includegraphics command and options

% \usepackage[parfill]{parskip} % Activate to begin paragraphs with an empty line rather than an indent
\usepackage[spanish]{babel}
%%% PACKAGES
\usepackage{amsmath,amssymb}
\usepackage{booktabs} % for much better looking tables
\usepackage{array} % for better arrays (eg matrices) in maths
\usepackage{paralist} % very flexible & customisable lists (eg. enumerate/itemize, etc.)
\usepackage{verbatim} % adds environment for commenting out blocks of text & for better verbatim
\usepackage{subfig} % make it possible to include more than one captioned figure/table in a single float
\usepackage{listings}
% These packages are all incorporated in the memoir class to one degree or another...

%%% HEADERS & FOOTERS
\usepackage{fancyhdr} % This should be set AFTER setting up the page geometry
\pagestyle{fancy} % options: empty , plain , fancy
\renewcommand{\headrulewidth}{0pt} % customise the layout...
\lhead{}\chead{}\rhead{}
\lfoot{}\cfoot{\thepage}\rfoot{}

%%% SECTION TITLE APPEARANCE
\usepackage{sectsty}
\allsectionsfont{\sffamily\mdseries\upshape} % (See the fntguide.pdf for font help)
% (This matches ConTeXt defaults)

%%% ToC (table of contents) APPEARANCE
\usepackage[nottoc,notlof,notlot]{tocbibind} % Put the bibliography in the ToC
\usepackage[titles,subfigure]{tocloft} % Alter the style of the Table of Contents
\renewcommand{\cftsecfont}{\rmfamily\mdseries\upshape}
\renewcommand{\cftsecpagefont}{\rmfamily\mdseries\upshape} % No bold!

%%% END Article customizations

%%% The "real" document content comes below...

\title{Taller Segundo Corte}
\author{Programación en R-Universidad Piloto de Colombia}
\date{} 
\begin{document}
\maketitle
\begin{enumerate}
\item Construya un vector con 10000 números aleatorios entre 0 y 9, cree un "contador" que calcula la frecuencia de ocurrencia de cada número del 0 al 9, y los consigne en un vector
\item Escriba un programa que pida al usuario ingresar un número entero, y que imprima "par" si el número es par e "impar" si el número es impar. Agregue a su programa un código que genere una advertencia en caso de que el usuario ingrese algo diferente a un número entero: \"Error. El usuario debe ingresar un número entero.\" (Investigue por su cuenta cómo lograr dicha validación y la generación del mensaje.)\par
\item Escriba un for loop que imprima todos los múltiplos de 3 desde 40 hasta 0 en orden decreciente. Esto es, 39, 36, 33,..., 3, 0.
\item Escriba un loop que imprima todos los números entre 6 y 30 que no son divisibles por 2, 3 o 5.
\item Escriba un programa llamado "Adivine ni número". El computador generará aleatoriamente un entero entre 1 y 100. El usuario digita un número y el computador responde "Menor" si el número aleatorio es menor que el escogido por el usuario, "Mayor" si el número aleatorio es mayor, y "¡Correcto!" si el usuario adivina el número. El jugador puede continuar ingresando números hasta que adivine correctamente.\par
Ejemplo:\par
- El número aleatorio es 79.\par
- El computador muestra el texto "Adivine el número entre 1 y 100:" y espera a que el usuario lo digite.\par
- El usuario digita el número que está abajo en itálicas.\par
- El computador devuelve uno de tres textos, según el caso: "Mayor", "Menor", o "¡Correcto!".\par
Adivine el número entre 1 y 100: 40 Mayor\par
Adivine el número entre 1 y 100: 70 Mayor\par
Adivine el número entre 1 y 100:80 Menor\par
Adivine el número entre 1 y 100: 77 Mayor\par
Adivine el número entre 1 y 100: 79 ¡Correcto!\par
\item Escriba una función que agregue un elemento a un vector\par
\item Escríba una función que ordene (de forma ascedente y descendente) un vector según sus valores.\par
\item Escriba un programa que concatene tres vectores en uno nuevo\par
\item Escriba una función que tome un vector a y arroje un nuevo vector con solo los elementos pares de a.\par
\item Escriba una función que arroje los valores mínimo y máximo de un vector \par
\item Escriba una función que tome un vector y multiplique todos sus elementos\par
\item Use la función scan() para convertir en dataframe el archivo Intima\_Media3.txt, es necesario primero leer el archivo
\item Importe los datos del archivo nutrition\_elderly.txt que contiene 13 variables medidas sobre 226 individuos con scan() o readLines() y obtenga el dataframe correspondiente
\item Importe los datos del archivo Birth\_weight.txt que contiene 10 variables medidas sobre 189 individuos con scan() o readLines() y obtenga el dataframe correspondiente
\end{enumerate}
\end{document}
