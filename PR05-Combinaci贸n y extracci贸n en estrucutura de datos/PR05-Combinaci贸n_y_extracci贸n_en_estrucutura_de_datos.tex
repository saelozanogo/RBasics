\documentclass[ignorenonframetext,]{beamer}
\setbeamertemplate{caption}[numbered]
\setbeamertemplate{caption label separator}{: }
\setbeamercolor{caption name}{fg=normal text.fg}
\beamertemplatenavigationsymbolsempty
\usepackage{lmodern}
\usepackage{amssymb,amsmath}
\usepackage{ifxetex,ifluatex}
\usepackage{fixltx2e} % provides \textsubscript
\ifnum 0\ifxetex 1\fi\ifluatex 1\fi=0 % if pdftex
  \usepackage[T1]{fontenc}
  \usepackage[utf8]{inputenc}
\else % if luatex or xelatex
  \ifxetex
    \usepackage{mathspec}
  \else
    \usepackage{fontspec}
  \fi
  \defaultfontfeatures{Ligatures=TeX,Scale=MatchLowercase}
\fi
\usetheme[]{Darmstadt}
\usefonttheme{structurebold}
% use upquote if available, for straight quotes in verbatim environments
\IfFileExists{upquote.sty}{\usepackage{upquote}}{}
% use microtype if available
\IfFileExists{microtype.sty}{%
\usepackage{microtype}
\UseMicrotypeSet[protrusion]{basicmath} % disable protrusion for tt fonts
}{}
\newif\ifbibliography
\hypersetup{
            pdftitle={Combinación y extracción en estructuras de datos},
            pdfauthor={Santiago Lozano},
            pdfborder={0 0 0},
            breaklinks=true}
\urlstyle{same}  % don't use monospace font for urls
\usepackage{color}
\usepackage{fancyvrb}
\newcommand{\VerbBar}{|}
\newcommand{\VERB}{\Verb[commandchars=\\\{\}]}
\DefineVerbatimEnvironment{Highlighting}{Verbatim}{commandchars=\\\{\}}
% Add ',fontsize=\small' for more characters per line
\usepackage{framed}
\definecolor{shadecolor}{RGB}{248,248,248}
\newenvironment{Shaded}{\begin{snugshade}}{\end{snugshade}}
\newcommand{\KeywordTok}[1]{\textcolor[rgb]{0.13,0.29,0.53}{\textbf{#1}}}
\newcommand{\DataTypeTok}[1]{\textcolor[rgb]{0.13,0.29,0.53}{#1}}
\newcommand{\DecValTok}[1]{\textcolor[rgb]{0.00,0.00,0.81}{#1}}
\newcommand{\BaseNTok}[1]{\textcolor[rgb]{0.00,0.00,0.81}{#1}}
\newcommand{\FloatTok}[1]{\textcolor[rgb]{0.00,0.00,0.81}{#1}}
\newcommand{\ConstantTok}[1]{\textcolor[rgb]{0.00,0.00,0.00}{#1}}
\newcommand{\CharTok}[1]{\textcolor[rgb]{0.31,0.60,0.02}{#1}}
\newcommand{\SpecialCharTok}[1]{\textcolor[rgb]{0.00,0.00,0.00}{#1}}
\newcommand{\StringTok}[1]{\textcolor[rgb]{0.31,0.60,0.02}{#1}}
\newcommand{\VerbatimStringTok}[1]{\textcolor[rgb]{0.31,0.60,0.02}{#1}}
\newcommand{\SpecialStringTok}[1]{\textcolor[rgb]{0.31,0.60,0.02}{#1}}
\newcommand{\ImportTok}[1]{#1}
\newcommand{\CommentTok}[1]{\textcolor[rgb]{0.56,0.35,0.01}{\textit{#1}}}
\newcommand{\DocumentationTok}[1]{\textcolor[rgb]{0.56,0.35,0.01}{\textbf{\textit{#1}}}}
\newcommand{\AnnotationTok}[1]{\textcolor[rgb]{0.56,0.35,0.01}{\textbf{\textit{#1}}}}
\newcommand{\CommentVarTok}[1]{\textcolor[rgb]{0.56,0.35,0.01}{\textbf{\textit{#1}}}}
\newcommand{\OtherTok}[1]{\textcolor[rgb]{0.56,0.35,0.01}{#1}}
\newcommand{\FunctionTok}[1]{\textcolor[rgb]{0.00,0.00,0.00}{#1}}
\newcommand{\VariableTok}[1]{\textcolor[rgb]{0.00,0.00,0.00}{#1}}
\newcommand{\ControlFlowTok}[1]{\textcolor[rgb]{0.13,0.29,0.53}{\textbf{#1}}}
\newcommand{\OperatorTok}[1]{\textcolor[rgb]{0.81,0.36,0.00}{\textbf{#1}}}
\newcommand{\BuiltInTok}[1]{#1}
\newcommand{\ExtensionTok}[1]{#1}
\newcommand{\PreprocessorTok}[1]{\textcolor[rgb]{0.56,0.35,0.01}{\textit{#1}}}
\newcommand{\AttributeTok}[1]{\textcolor[rgb]{0.77,0.63,0.00}{#1}}
\newcommand{\RegionMarkerTok}[1]{#1}
\newcommand{\InformationTok}[1]{\textcolor[rgb]{0.56,0.35,0.01}{\textbf{\textit{#1}}}}
\newcommand{\WarningTok}[1]{\textcolor[rgb]{0.56,0.35,0.01}{\textbf{\textit{#1}}}}
\newcommand{\AlertTok}[1]{\textcolor[rgb]{0.94,0.16,0.16}{#1}}
\newcommand{\ErrorTok}[1]{\textcolor[rgb]{0.64,0.00,0.00}{\textbf{#1}}}
\newcommand{\NormalTok}[1]{#1}

% Prevent slide breaks in the middle of a paragraph:
\widowpenalties 1 10000
\raggedbottom

\AtBeginPart{
  \let\insertpartnumber\relax
  \let\partname\relax
  \frame{\partpage}
}
\AtBeginSection{
  \ifbibliography
  \else
    \let\insertsectionnumber\relax
    \let\sectionname\relax
    \frame{\sectionpage}
  \fi
}
\AtBeginSubsection{
  \let\insertsubsectionnumber\relax
  \let\subsectionname\relax
  \frame{\subsectionpage}
}

\setlength{\parindent}{0pt}
\setlength{\parskip}{6pt plus 2pt minus 1pt}
\setlength{\emergencystretch}{3em}  % prevent overfull lines
\providecommand{\tightlist}{%
  \setlength{\itemsep}{0pt}\setlength{\parskip}{0pt}}
\setcounter{secnumdepth}{0}

\title{Combinación y extracción en estructuras de datos}
\author{Santiago Lozano}
\date{28 de febrero de 2020}

\begin{document}
\frame{\titlepage}

\begin{frame}[fragile]{Combinación de matrices o data.frames}

Para concatenar matrices o data.frames usamos las funciones cbind() para
concatenar columnar y rbind() para concatenar filas 

\begin{Shaded}
\begin{Highlighting}[]
\KeywordTok{cbind}\NormalTok{(}\DecValTok{1}\OperatorTok{:}\DecValTok{4}\NormalTok{,}\DecValTok{5}\OperatorTok{:}\DecValTok{8}\NormalTok{)}
\end{Highlighting}
\end{Shaded}

\begin{verbatim}
##      [,1] [,2]
## [1,]    1    5
## [2,]    2    6
## [3,]    3    7
## [4,]    4    8
\end{verbatim}

\end{frame}

\begin{frame}[fragile]{Combinación de matrices o data.frames}

Sin embargo esta función no es óptima en el siguiente ejemplo

\begin{Shaded}
\begin{Highlighting}[]
\NormalTok{X1 <-}\StringTok{ }\KeywordTok{data.frame}\NormalTok{(}\DataTypeTok{Id=}\DecValTok{1}\OperatorTok{:}\DecValTok{4}\NormalTok{,}\DataTypeTok{Genero=}\KeywordTok{c}\NormalTok{(}\StringTok{"M"}\NormalTok{,}\StringTok{"F"}\NormalTok{,}\StringTok{"F"}\NormalTok{,}\StringTok{"M"}\NormalTok{),}
                 \DataTypeTok{Peso=}\KeywordTok{c}\NormalTok{(}\DecValTok{75}\NormalTok{,}\DecValTok{68}\NormalTok{,}\DecValTok{48}\NormalTok{,}\DecValTok{72}\NormalTok{))}
\NormalTok{X1}
\end{Highlighting}
\end{Shaded}

\begin{verbatim}
##   Id Genero Peso
## 1  1      M   75
## 2  2      F   68
## 3  3      F   48
## 4  4      M   72
\end{verbatim}

\end{frame}

\begin{frame}[fragile]{Combinación de matrices o data.frames}

\begin{Shaded}
\begin{Highlighting}[]
\NormalTok{X2 <-}\StringTok{ }\KeywordTok{data.frame}\NormalTok{(}\DataTypeTok{Id=}\DecValTok{1}\OperatorTok{:}\DecValTok{4}\NormalTok{,}\DataTypeTok{Genero=}\KeywordTok{c}\NormalTok{(}\StringTok{"M"}\NormalTok{,}\StringTok{"F"}\NormalTok{,}\StringTok{"F"}\NormalTok{,}\StringTok{"M"}\NormalTok{),}
                 \DataTypeTok{Altura=}\KeywordTok{c}\NormalTok{(}\DecValTok{182}\NormalTok{,}\DecValTok{165}\NormalTok{,}\DecValTok{160}\NormalTok{,}\DecValTok{178}\NormalTok{))}
\NormalTok{X2}
\end{Highlighting}
\end{Shaded}

\begin{verbatim}
##   Id Genero Altura
## 1  1      M    182
## 2  2      F    165
## 3  3      F    160
## 4  4      M    178
\end{verbatim}

\end{frame}

\begin{frame}[fragile]{Combinación de matrices o data.frames}

\begin{Shaded}
\begin{Highlighting}[]
\KeywordTok{cbind}\NormalTok{(X1,X2)}
\end{Highlighting}
\end{Shaded}

\begin{verbatim}
##   Id Genero Peso Id Genero Altura
## 1  1      M   75  1      M    182
## 2  2      F   68  2      F    165
## 3  3      F   48  3      F    160
## 4  4      M   72  4      M    178
\end{verbatim}

\end{frame}

\begin{frame}[fragile]{Combinación de matrices o data.frames}

Para estos casos la función que permite combinar elementos en bases de
datos es merge()

\begin{Shaded}
\begin{Highlighting}[]
\KeywordTok{merge}\NormalTok{(X1,X2)}
\end{Highlighting}
\end{Shaded}

\begin{verbatim}
##   Id Genero Peso Altura
## 1  1      M   75    182
## 2  2      F   68    165
## 3  3      F   48    160
## 4  4      M   72    178
\end{verbatim}

\end{frame}

\begin{frame}[fragile]{Combinación de matrices o data.frames}

Ahora suonga que tenemos

\begin{Shaded}
\begin{Highlighting}[]
\NormalTok{X3 <-}\StringTok{ }\KeywordTok{data.frame}\NormalTok{(}\DataTypeTok{Id=}\KeywordTok{c}\NormalTok{(}\DecValTok{2}\NormalTok{,}\DecValTok{1}\NormalTok{,}\DecValTok{4}\NormalTok{,}\DecValTok{3}\NormalTok{),}\DataTypeTok{Genero=}\KeywordTok{c}\NormalTok{(}\StringTok{"F"}\NormalTok{,}\StringTok{"M"}\NormalTok{,}\StringTok{"M"}\NormalTok{,}\StringTok{"F"}\NormalTok{),}
                 \DataTypeTok{Altura=}\KeywordTok{c}\NormalTok{(}\DecValTok{165}\NormalTok{,}\DecValTok{182}\NormalTok{,}\DecValTok{178}\NormalTok{,}\DecValTok{160}\NormalTok{))}
\NormalTok{X3}
\end{Highlighting}
\end{Shaded}

\begin{verbatim}
##   Id Genero Altura
## 1  2      F    165
## 2  1      M    182
## 3  4      M    178
## 4  3      F    160
\end{verbatim}

\end{frame}

\begin{frame}[fragile]{Combinación de matrices o data.frames}

de igual forma merge() es la función adecuada para combinar X1 y X3

\begin{Shaded}
\begin{Highlighting}[]
\KeywordTok{merge}\NormalTok{(X1,X3)}
\end{Highlighting}
\end{Shaded}

\begin{verbatim}
##   Id Genero Peso Altura
## 1  1      M   75    182
## 2  2      F   68    165
## 3  3      F   48    160
## 4  4      M   72    178
\end{verbatim}

\end{frame}

\begin{frame}{Combinación de matrices o data.frames}

Como hemos visto la función merge() combinados data.frames. La
combinación está basada en las columnas de esos data.frames que tienen
los mismos nombres. Las cuales se denominarán ``Columnas comunes''. El
argumento \textbf{by} puede ser usado para forzar cuales columnas son
comunes. El valor de este argumento puede ser un vector de nombres,
vector de índices o vectores lógicos y las otras colunas son tratadas
como distintas. La función merge() trbaja de la siguiente manera,
entonces sea X y Y dos data.frames y veamos como trabaja

\begin{itemize}
\tightlist
\item
  Toda fila de un del data.frame X, en la función merge() compara los
  elementos de esta fila con respecto a todas las filas de Y, pero solo
  sobre el subconjunto de columnas en común
\end{itemize}

\end{frame}

\begin{frame}{Combinación de matrices o data.frames}

\begin{itemize}
\item
  si encuentra un match perfect, considera que es un mismo individuo:
  este individuo es agregado al nuevo data.frame resultado de la
  combinación y completa con los valores de las columnas no comunes de X
  y Y
\item
  Si no hay match perfecto, el individuo es igualmente agregado al
  data.frame resultante con NA's (si el argumento all() toma el valor
  TRUE) o removido (si el argumento all() es FALSE)
\item
  La operación se repite a la segunda fila y así sucesivamente
\end{itemize}

\end{frame}

\begin{frame}[fragile]{Combinación de matrices o data.frames}

Veamos un ejemplo

\begin{Shaded}
\begin{Highlighting}[]
\NormalTok{X <-}\StringTok{ }\KeywordTok{data.frame}\NormalTok{(}\DataTypeTok{Genero=}\KeywordTok{c}\NormalTok{(}\StringTok{"F"}\NormalTok{,}\StringTok{"M"}\NormalTok{,}\StringTok{"M"}\NormalTok{,}\StringTok{"F"}\NormalTok{),}
                \DataTypeTok{Altura=}\KeywordTok{c}\NormalTok{(}\DecValTok{165}\NormalTok{, }\DecValTok{82}\NormalTok{,}\DecValTok{178}\NormalTok{,}\DecValTok{160}\NormalTok{),}
                \DataTypeTok{Peso=}\KeywordTok{c}\NormalTok{(}\DecValTok{50}\NormalTok{,}\DecValTok{65}\NormalTok{,}\DecValTok{67}\NormalTok{,}\DecValTok{55}\NormalTok{),}
                \DataTypeTok{Ingreso=}\KeywordTok{c}\NormalTok{(}\DecValTok{80}\NormalTok{,}\DecValTok{90}\NormalTok{,}\DecValTok{60}\NormalTok{,}\DecValTok{50}\NormalTok{))}
\NormalTok{X}
\end{Highlighting}
\end{Shaded}

\begin{verbatim}
##   Genero Altura Peso Ingreso
## 1      F    165   50      80
## 2      M     82   65      90
## 3      M    178   67      60
## 4      F    160   55      50
\end{verbatim}

\end{frame}

\begin{frame}[fragile]{Combinación de matrices o data.frames}

\begin{Shaded}
\begin{Highlighting}[]
\NormalTok{Y <-}\StringTok{ }\KeywordTok{data.frame}\NormalTok{(}\DataTypeTok{Genero=}\KeywordTok{c}\NormalTok{(}\StringTok{"F"}\NormalTok{,}\StringTok{"M"}\NormalTok{,}\StringTok{"M"}\NormalTok{,}\StringTok{"F"}\NormalTok{),}
                \DataTypeTok{Altura=}\KeywordTok{c}\NormalTok{(}\DecValTok{165}\NormalTok{, }\DecValTok{82}\NormalTok{,}\DecValTok{178}\NormalTok{,}\DecValTok{160}\NormalTok{),}
                \DataTypeTok{Peso=}\KeywordTok{c}\NormalTok{(}\DecValTok{55}\NormalTok{,}\DecValTok{65}\NormalTok{,}\DecValTok{67}\NormalTok{,}\DecValTok{85}\NormalTok{),}
                \DataTypeTok{Salario=}\KeywordTok{c}\NormalTok{(}\DecValTok{70}\NormalTok{,}\DecValTok{90}\NormalTok{,}\DecValTok{40}\NormalTok{,}\DecValTok{40}\NormalTok{) ,}
                \DataTypeTok{row.names=}\DecValTok{4}\OperatorTok{:}\DecValTok{7}\NormalTok{)}
\NormalTok{Y}
\end{Highlighting}
\end{Shaded}

\begin{verbatim}
##   Genero Altura Peso Salario
## 4      F    165   55      70
## 5      M     82   65      90
## 6      M    178   67      40
## 7      F    160   85      40
\end{verbatim}

\end{frame}

\begin{frame}[fragile]{Combinación de matrices o data.frames}

\begin{Shaded}
\begin{Highlighting}[]
\KeywordTok{merge}\NormalTok{(X,Y,}\DataTypeTok{by=}\KeywordTok{c}\NormalTok{(}\StringTok{"Genero"}\NormalTok{,}\StringTok{"Peso"}\NormalTok{))}
\end{Highlighting}
\end{Shaded}

\begin{verbatim}
##   Genero Peso Altura.x Ingreso Altura.y Salario
## 1      F   55      160      50      165      70
## 2      M   65       82      90       82      90
## 3      M   67      178      60      178      40
\end{verbatim}

\end{frame}

\begin{frame}[fragile]{Combinación de matrices o data.frames}

\begin{Shaded}
\begin{Highlighting}[]
\KeywordTok{merge}\NormalTok{(X,Y,}\DataTypeTok{by=}\KeywordTok{c}\NormalTok{(}\StringTok{"Genero"}\NormalTok{,}\StringTok{"Peso"}\NormalTok{),}\DataTypeTok{all=}\OtherTok{TRUE}\NormalTok{)}
\end{Highlighting}
\end{Shaded}

\begin{verbatim}
##   Genero Peso Altura.x Ingreso Altura.y Salario
## 1      F   50      165      80       NA      NA
## 2      F   55      160      50      165      70
## 3      F   85       NA      NA      160      40
## 4      M   65       82      90       82      90
## 5      M   67      178      60      178      40
\end{verbatim}

\end{frame}

\begin{frame}[fragile]{Combinación de matrices o data.frames}

Tengamos en cuenta que la función merge() no tiene en cuenta los nombres
de los individuos que asigna R, cuando determina los individuos comunes.
Los nombres pueden ser incluidos agregando una columna Id a los dos
data.frames, para identificar los individuos usando ``row.names'' como
el valor en el argumento by

\begin{Shaded}
\begin{Highlighting}[]
\KeywordTok{merge}\NormalTok{(X,Y,}\DataTypeTok{by=}\KeywordTok{c}\NormalTok{(}\StringTok{"row.names"}\NormalTok{,}\StringTok{"Peso"}\NormalTok{))}
\end{Highlighting}
\end{Shaded}

\begin{verbatim}
##   Row.names Peso Genero.x Altura.x Ingreso Genero.y 
## 1         4   55        F      160      50        F     
## Altura.y Salario
##     165      70

\end{verbatim}

\end{frame}

\begin{frame}[fragile]{Combinación de matrices o data.frames}

\begin{Shaded}
\begin{Highlighting}[]
\KeywordTok{merge}\NormalTok{(X,Y,}\DataTypeTok{by=}\KeywordTok{c}\NormalTok{(}\StringTok{"row.names"}\NormalTok{,}\StringTok{"Peso"}\NormalTok{),}\DataTypeTok{all=}\OtherTok{TRUE}\NormalTok{)}
\end{Highlighting}
\end{Shaded}

\begin{verbatim}
##   Row.names Peso Genero.x Altura.x Ingreso Genero.y Altura.y Salario
## 1         1   50        F      165      80     <NA>       NA      NA
## 2         2   65        M       82      90     <NA>       NA      NA
## 3         3   67        M      178      60     <NA>       NA      NA
## 4         4   55        F      160      50        F      165      70
## 5         5   65     <NA>       NA      NA        M       82      90
## 6         6   67     <NA>       NA      NA        M      178      40
## 7         7   85     <NA>       NA      NA        F      160      40
\end{verbatim}


\end{frame}

\begin{frame}[fragile]{Combinación de matrices o data.frames}

\begin{verbatim}
##    Altura.y Salario
## 1        NA      NA
## 2        NA      NA
## 3        NA      NA
## 4       165      70
## 5        82      90
## 6       178      40
## 7       160      40
\end{verbatim}


\end{frame}


\begin{frame}[fragile]{Combinación de matrices o data.frames}

Para concatenar filas usamos rbind()

\begin{Shaded}
\begin{Highlighting}[]
\KeywordTok{rbind}\NormalTok{(}\DecValTok{1}\OperatorTok{:}\DecValTok{4}\NormalTok{,}\DecValTok{5}\OperatorTok{:}\DecValTok{8}\NormalTok{)}
\end{Highlighting}
\end{Shaded}

\begin{verbatim}
##      [,1] [,2] [,3] [,4]
## [1,]    1    2    3    4
## [2,]    5    6    7    8
\end{verbatim}

\end{frame}

\begin{frame}[fragile]{Combinación de matrices o data.frames}

veamos una función útil en algunos casos

\begin{Shaded}
\begin{Highlighting}[]
\NormalTok{df1 <-}\StringTok{ }\KeywordTok{data.frame}\NormalTok{(}\DataTypeTok{A=}\DecValTok{1}\OperatorTok{:}\DecValTok{5}\NormalTok{, }\DataTypeTok{B=}\NormalTok{LETTERS[}\DecValTok{1}\OperatorTok{:}\DecValTok{5}\NormalTok{])}
\NormalTok{df1}
\end{Highlighting}
\end{Shaded}

\begin{verbatim}
##   A B
## 1 1 A
## 2 2 B
## 3 3 C
## 4 4 D
## 5 5 E
\end{verbatim}

\end{frame}

\begin{frame}[fragile]{Combinación de matrices o data.frames}

\begin{Shaded}
\begin{Highlighting}[]
\NormalTok{df2 <-}\StringTok{ }\KeywordTok{data.frame}\NormalTok{(}\DataTypeTok{A=}\DecValTok{6}\OperatorTok{:}\DecValTok{10}\NormalTok{, }\DataTypeTok{E=}\NormalTok{letters[}\DecValTok{1}\OperatorTok{:}\DecValTok{5}\NormalTok{])}
\NormalTok{df2}
\end{Highlighting}
\end{Shaded}

\begin{verbatim}
##    A E
## 1  6 a
## 2  7 b
## 3  8 c
## 4  9 d
## 5 10 e
\end{verbatim}

\end{frame}

\begin{frame}[fragile]{Combinación de matrices o data.frames}

\begin{Shaded}
\begin{Highlighting}[]
\KeywordTok{smartbind}\NormalTok{(df1, df2)}
\end{Highlighting}
\end{Shaded}

\begin{verbatim}
##      A    B    E
## 1:1  1    A <NA>
## 1:2  2    B <NA>
## 1:3  3    C <NA>
## 1:4  4    D <NA>
## 1:5  5    E <NA>
## 2:1  6 <NA>    a
## 2:2  7 <NA>    b
## 2:3  8 <NA>    c
## 2:4  9 <NA>    d
## 2:5 10 <NA>    e
\end{verbatim}

\end{frame}

\begin{frame}{Operaciones sobre listas}

Las funciones lapply() y sapply() son similares a la función apply(),
aplicando una función a todos los elementos de una lista, el output
corresponde a una lista, y cada elemento de la lista genera un vector si
es posible

\end{frame}

\begin{frame}[fragile]{Operaciones sobre listas}

\begin{Shaded}
\begin{Highlighting}[]
\NormalTok{x <-}\StringTok{ }\KeywordTok{list}\NormalTok{(}\DataTypeTok{a =} \DecValTok{1}\OperatorTok{:}\DecValTok{10}\NormalTok{, }\DataTypeTok{beta =} \KeywordTok{exp}\NormalTok{(}\OperatorTok{-}\DecValTok{3}\OperatorTok{:}\DecValTok{3}\NormalTok{),}
          \DataTypeTok{logic =} \KeywordTok{c}\NormalTok{(}\OtherTok{TRUE}\NormalTok{,}\OtherTok{FALSE}\NormalTok{,}\OtherTok{FALSE}\NormalTok{,}\OtherTok{TRUE}\NormalTok{))}
\NormalTok{x}
\end{Highlighting}
\end{Shaded}

\begin{verbatim}
## $a
##  [1]  1  2  3  4  5  6  7  8  9 10
## 
## $beta
## [1]  0.04978707  0.13533528  0.36787944  1.00000000 
## [5]  2.71828183  7.38905610  2.71828183  7.38905610 
## [9] 20.08553692
## 
## $logic
## [1]  TRUE FALSE FALSE  TRUE
\end{verbatim}

\end{frame}

\begin{frame}[fragile]{Operaciones sobre listas}

\begin{Shaded}
\begin{Highlighting}[]
\KeywordTok{lapply}\NormalTok{(x,mean)}
\end{Highlighting}
\end{Shaded}

\begin{verbatim}
## $a
## [1] 5.5
## 
## $beta
## [1] 4.535125
## 
## $logic
## [1] 0.5
\end{verbatim}

\end{frame}

\begin{frame}[fragile]{Operaciones sobre listas}

\begin{Shaded}
\begin{Highlighting}[]
\KeywordTok{lapply}\NormalTok{(x,quantile,}\DataTypeTok{probs=}\NormalTok{(}\DecValTok{1}\OperatorTok{:}\DecValTok{3}\NormalTok{)}\OperatorTok{/}\DecValTok{4}\NormalTok{)}
\end{Highlighting}
\end{Shaded}

\begin{verbatim}
## $a
##  25%  50%  75% 
## 3.25 5.50 7.75 
## 
## $beta
##       25%       50%       75% 
## 0.2516074 1.0000000 5.0536690 
## 
## $logic
## 25% 50% 75% 
## 0.0 0.5 1.0
\end{verbatim}

\end{frame}

\begin{frame}[fragile]{Operaciones sobre listas}

\begin{Shaded}
\begin{Highlighting}[]
\KeywordTok{sapply}\NormalTok{(x, quantile)}
\end{Highlighting}
\end{Shaded}

\begin{verbatim}
##          a        beta logic
## 0%    1.00  0.04978707   0.0
## 25%   3.25  0.25160736   0.0
## 50%   5.50  1.00000000   0.5
## 75%   7.75  5.05366896   1.0
## 100% 10.00 20.08553692   1.0
\end{verbatim}

\end{frame}

\begin{frame}{Operaciones sobre listas}

Si desea aplicar una función a un vector (en lugar de al margen de una
matriz), use sapply(. Aquí está el código para generar una lista de
secuencias.

\end{frame}

\begin{frame}[fragile]{Operaciones sobre listas}

\begin{Shaded}
\begin{Highlighting}[]
\NormalTok{i36 <-}\StringTok{ }\KeywordTok{sapply}\NormalTok{(}\DecValTok{3}\OperatorTok{:}\DecValTok{6}\NormalTok{, seq)}\CommentTok{# Crea una lista de vectores}
\NormalTok{i36}
\end{Highlighting}
\end{Shaded}

\begin{verbatim}
## [[1]]
## [1] 1 2 3
## 
## [[2]]
## [1] 1 2 3 4
## 
## [[3]]
## [1] 1 2 3 4 5
## 
## [[4]]
## [1] 1 2 3 4 5 6
\end{verbatim}

\end{frame}

\begin{frame}[fragile]{Operaciones sobre listas}

\begin{Shaded}
\begin{Highlighting}[]
\KeywordTok{sapply}\NormalTok{(i36, sum) }
\end{Highlighting}
\end{Shaded}

\begin{verbatim}
## [1]  6 10 15 21
\end{verbatim}

\end{frame}

\begin{frame}[fragile]{Ejercicios (1)}
\begin{Shaded}
\begin{Highlighting}[]
\KeywordTok{setwd}\NormalTok{(}\StringTok{"~/Progrmación en R/2020-I/PR05-}
	\StringTok{"Combinación y extracción en estrucutura de datos"}
	\StringTok{"de datos"}\NormalTok{)}
\NormalTok{(matdata <-}\StringTok{ }\KeywordTok{read.table}\NormalTok{(}\StringTok{sweepdata.txt"}\NormalTok{))}
\end{Highlighting}
\end{Shaded}
\begin{verbatim}
##    V1 V2  V3  V4
## 1   3 12 0.4 125
## 2   5 12 0.7 166
## 3   7 15 0.8 174
## 4   7 14 0.7 128
## 5   5 18 0.3 136
## 6   9 13 0.2 155
## 7   7 15 0.5 115
## 8   2 13 0.5 169
## 9   1 10 0.1 182
## 10  0 11 0.2 166
\end{verbatim}

\end{frame}

\begin{frame}[fragile]{Ejercicios (1)}

En este ejemplo, queremos expresar una matriz en términos de las
desviaciones de cada valor de su media de columna. Primero aplique la
función apply() para sacar la media a cada una de sus columnas

\begin{Shaded}
\begin{Highlighting}[]
\NormalTok{(cols <-}\StringTok{ }\KeywordTok{apply}\NormalTok{(matdata,}\DataTypeTok{MARGIN=}\DecValTok{2}\NormalTok{,mean))}
\end{Highlighting}
\end{Shaded}

\begin{verbatim}
##     V1     V2     V3     V4 
##   4.60  13.30   0.44 151.60
\end{verbatim}

\end{frame}

\begin{frame}[fragile]{Ejercicios (1)}

Ahora a cada valor de la columna le resto la media correspondiente
usando sweep()

\begin{Shaded}
\begin{Highlighting}[]
\KeywordTok{sweep}\NormalTok{(matdata,}\DataTypeTok{MARGIN=}\DecValTok{2}\NormalTok{,}\DataTypeTok{STATS=}\NormalTok{cols,}\DataTypeTok{FUN=}\StringTok{"-"}\NormalTok{)}
\end{Highlighting}
\end{Shaded}

\begin{verbatim}
##      V1   V2    V3    V4
## 1  -1.6 -1.3 -0.04 -26.6
## 2   0.4 -1.3  0.26  14.4
## 3   2.4  1.7  0.36  22.4
## 4   2.4  0.7  0.26 -23.6
## 5   0.4  4.7 -0.14 -15.6
## 6   4.4 -0.3 -0.24   3.4
## 7   2.4  1.7  0.06 -36.6
## 8  -2.6 -0.3  0.06  17.4
## 9  -3.6 -3.3 -0.34  30.4
## 10 -4.6 -2.3 -0.24  14.4
\end{verbatim}

\end{frame}

\begin{frame}[fragile]{Ejercicios (2)}

Suponga que desea obtener los subíndices para un barrido de datos en
columnas o en filas. Establezca los subíndices de fila repetidos en cada
columna usando apply():

\begin{Shaded}
\begin{Highlighting}[]
\KeywordTok{apply}\NormalTok{(matdata,}\DataTypeTok{MARGIN=}\DecValTok{2}\NormalTok{,}\DataTypeTok{FUN=}\ControlFlowTok{function}\NormalTok{ (x) }\DecValTok{1}\OperatorTok{:}\DecValTok{10}\NormalTok{)}
\end{Highlighting}
\end{Shaded}

\begin{verbatim}
##       V1 V2 V3 V4
##  [1,]  1  1  1  1
##  [2,]  2  2  2  2
##  [3,]  3  3  3  3
##  [4,]  4  4  4  4
##  [5,]  5  5  5  5
##  [6,]  6  6  6  6
##  [7,]  7  7  7  7
##  [8,]  8  8  8  8
##  [9,]  9  9  9  9
## [10,] 10 10 10 10
\end{verbatim}

\end{frame}

\begin{frame}[fragile]{Ejercicios (2)}

y para las columnas usando apply()

\begin{Shaded}
\begin{Highlighting}[]
\KeywordTok{t}\NormalTok{(}\KeywordTok{apply}\NormalTok{(matdata,}\DataTypeTok{MARGIN=}\DecValTok{1}\NormalTok{,}\DataTypeTok{FUN=}\ControlFlowTok{function}\NormalTok{ (x) }\DecValTok{1}\OperatorTok{:}\DecValTok{4}\NormalTok{))}
\end{Highlighting}
\end{Shaded}

\begin{verbatim}
##       [,1] [,2] [,3] [,4]
##  [1,]    1    2    3    4
##  [2,]    1    2    3    4
##  [3,]    1    2    3    4
##  [4,]    1    2    3    4
##  [5,]    1    2    3    4
##  [6,]    1    2    3    4
##  [7,]    1    2    3    4
##  [8,]    1    2    3    4
##  [9,]    1    2    3    4
## [10,]    1    2    3    4
\end{verbatim}

\end{frame}

\begin{frame}[fragile]{Ejercicios (3)}

Realice lo misimo con la función sweep()

\begin{Shaded}
\begin{Highlighting}[]
\KeywordTok{sweep}\NormalTok{(matdata,}\DataTypeTok{MARGIN=}\DecValTok{1}\NormalTok{,}\DataTypeTok{STATS=}\DecValTok{1}\OperatorTok{:}\DecValTok{10}\NormalTok{,}\DataTypeTok{FUN=}\ControlFlowTok{function}\NormalTok{(a,b) \{b\})}
\end{Highlighting}
\end{Shaded}

\begin{verbatim}
##       [,1] [,2] [,3] [,4]
##  [1,]    1    1    1    1
##  [2,]    2    2    2    2
##  [3,]    3    3    3    3
##  [4,]    4    4    4    4
##  [5,]    5    5    5    5
##  [6,]    6    6    6    6
##  [7,]    7    7    7    7
##  [8,]    8    8    8    8
##  [9,]    9    9    9    9
## [10,]   10   10   10   10
\end{verbatim}

\end{frame}

\begin{frame}[fragile]{Ejercicios (4)}

\begin{Shaded}
\begin{Highlighting}[]
\KeywordTok{sweep}\NormalTok{(matdata,}\DecValTok{2}\NormalTok{,}\DecValTok{1}\OperatorTok{:}\DecValTok{4}\NormalTok{,}\ControlFlowTok{function}\NormalTok{(a,b) b)}
\end{Highlighting}
\end{Shaded}

\begin{verbatim}
##       [,1] [,2] [,3] [,4]
##  [1,]    1    2    3    4
##  [2,]    1    2    3    4
##  [3,]    1    2    3    4
##  [4,]    1    2    3    4
##  [5,]    1    2    3    4
##  [6,]    1    2    3    4
##  [7,]    1    2    3    4
##  [8,]    1    2    3    4
##  [9,]    1    2    3    4
## [10,]    1    2    3    4
\end{verbatim}

\end{frame}

\begin{frame}[fragile]{Ejercicios (5)}

Tenga en cuenta que en ambos casos, la respuesta producida por apply()
es un vector en lugar de una matriz. Puede aplicar funciones a los
elementos individuales de la matriz en lugar de a los márgenes. El
margen que especifique solo influye en la forma de la matriz resultante.

\begin{Shaded}
\begin{Highlighting}[]
\NormalTok{(X <-}\StringTok{ }\KeywordTok{matrix}\NormalTok{(}\DecValTok{1}\OperatorTok{:}\DecValTok{24}\NormalTok{,}\DataTypeTok{nrow=}\DecValTok{4}\NormalTok{))}
\end{Highlighting}
\end{Shaded}

\begin{verbatim}
##      [,1] [,2] [,3] [,4] [,5] [,6]
## [1,]    1    5    9   13   17   21
## [2,]    2    6   10   14   18   22
## [3,]    3    7   11   15   19   23
## [4,]    4    8   12   16   20   24
\end{verbatim}

\end{frame}

\begin{frame}[fragile]{Ejercicios (5)}

Aplique raíz cuadrada a todos los elementos de la matriz y muestremelos
por filas

\begin{Shaded}
\begin{Highlighting}[]
\KeywordTok{apply}\NormalTok{(X,}\DataTypeTok{MARGIN=}\DecValTok{1}\NormalTok{,}\DataTypeTok{FUN=}\NormalTok{sqrt)}
\end{Highlighting}
\end{Shaded}

\begin{verbatim}
##          [,1]     [,2]     [,3]     [,4]
## [1,] 1.000000 1.414214 1.732051 2.000000
## [2,] 2.236068 2.449490 2.645751 2.828427
## [3,] 3.000000 3.162278 3.316625 3.464102
## [4,] 3.605551 3.741657 3.872983 4.000000
## [5,] 4.123106 4.242641 4.358899 4.472136
## [6,] 4.582576 4.690416 4.795832 4.898979
\end{verbatim}

\end{frame}

\begin{frame}[fragile]{Ejercicios (5)}

ahora por columnas

\begin{Shaded}
\begin{Highlighting}[]
\KeywordTok{apply}\NormalTok{(X,}\DecValTok{2}\NormalTok{,sqrt)}
\end{Highlighting}
\end{Shaded}

\begin{verbatim}
##          [,1]     [,2]     [,3]     [,4]     [,5]    
## [1,] 1.000000 2.236068 3.000000 3.605551 4.123106 
## [2,] 1.414214 2.449490 3.162278 3.741657 4.242641 
## [3,] 1.732051 2.645751 3.316625 3.872983 4.358899 
## [4,] 2.000000 2.828427 3.464102 4.000000 4.472136 

##             [,6]
## [1,] 4.582576
## [2,] 4.690416
## [3,] 4.795832
## [4,] 4.898979
\end{verbatim}

\end{frame}

\begin{frame}{Extracción e inserción de elementos}

Es esta sección veremos los diferentes recursos que tiene R para extraer
componentes de n vector, pues algunas veces quisieramos usar no todos
los contenidos de un vector. En R podemos usar la función ``{[}''(), o
podemos usar ``{[}{]}'' en el cual establecemos los subíndices que
queremos extraer ó podemos tomar los siguientes argumentos

\begin{itemize}
\item
  un vector de índices de elementos a extraer
\item
  un vector de índices de elementos a no extraer
\item
  un vector de valores lógicos indicando cuales elementos se extraen
\end{itemize}

\end{frame}

\begin{frame}[fragile]{Extracción e inserción de elementos en vectores}

\begin{Shaded}
\begin{Highlighting}[]
\NormalTok{vec <-}\StringTok{ }\KeywordTok{c}\NormalTok{(}\DecValTok{2}\NormalTok{,}\DecValTok{3}\NormalTok{,}\DecValTok{4}\NormalTok{,}\DecValTok{8}\NormalTok{,}\DecValTok{3}\NormalTok{)}
\NormalTok{vec}
\end{Highlighting}
\end{Shaded}
\pause
\begin{verbatim}
## [1] 2 3 4 8 3
\end{verbatim}

\begin{Shaded}
\begin{Highlighting}[]
\NormalTok{vec[}\DecValTok{2}\NormalTok{]}
\end{Highlighting}
\end{Shaded}
\pause
\begin{verbatim}
## [1] 3
\end{verbatim}

\end{frame}

\begin{frame}[fragile]{Extracción e inserción de elementos en vectores}

\begin{Shaded}
\begin{Highlighting}[]
\StringTok{"["}\NormalTok{(vec,}\DecValTok{2}\NormalTok{)}
\end{Highlighting}
\end{Shaded}
\pause
\begin{verbatim}
## [1] 3
\end{verbatim}

\begin{Shaded}
\begin{Highlighting}[]
\NormalTok{vec[}\OperatorTok{-}\DecValTok{2}\NormalTok{]}
\end{Highlighting}
\end{Shaded}
\pause
\begin{verbatim}
## [1] 2 4 8 3
\end{verbatim}

\end{frame}

\begin{frame}[fragile]{Extracción e inserción de elementos en vectores
(slicing)}

Podemos realizar un barrido para seleccionar una cantidad masiva de
subíndices a extraer, debe tenerse en cuente que R maneja los subíndices
empezando en 1

\begin{Shaded}
\begin{Highlighting}[]
\NormalTok{vec[}\DecValTok{2}\OperatorTok{:}\DecValTok{5}\NormalTok{]}
\end{Highlighting}
\end{Shaded}
\pause
\begin{verbatim}
## [1] 3 4 8 3
\end{verbatim}

\begin{Shaded}
\begin{Highlighting}[]
\NormalTok{vec[}\OperatorTok{-}\KeywordTok{c}\NormalTok{(}\DecValTok{1}\NormalTok{,}\DecValTok{5}\NormalTok{)]}
\end{Highlighting}
\end{Shaded}
\pause
\begin{verbatim}
## [1] 3 4 8
\end{verbatim}

\end{frame}

\begin{frame}[fragile]{Extracción e inserción de elementos en vectores
(slicing)}

\begin{Shaded}
\begin{Highlighting}[]
\NormalTok{vec[}\KeywordTok{c}\NormalTok{(T,F,F,T,T)]}
\end{Highlighting}
\end{Shaded}
\pause
\begin{verbatim}
## [1] 2 8 3
\end{verbatim}

\begin{Shaded}
\begin{Highlighting}[]
\NormalTok{vec}\OperatorTok{>}\DecValTok{4}
\end{Highlighting}
\end{Shaded}
\pause
\begin{verbatim}
## [1] FALSE FALSE FALSE  TRUE FALSE
\end{verbatim}

\begin{Shaded}
\begin{Highlighting}[]
\NormalTok{vec[vec}\OperatorTok{>}\DecValTok{4}\NormalTok{]}
\end{Highlighting}
\end{Shaded}
\pause
\begin{verbatim}
## [1] 8
\end{verbatim}

\end{frame}

\begin{frame}[fragile]{Extracción e inserción de elementos en vectores}

Es importante notar que la simplicidad sintáctica de una instrucción
como x{[}y\textgreater{}0{]}, el cual extrae de x todos los elementos
del subíndice i tales que \(y_i>0\)

\begin{Shaded}
\begin{Highlighting}[]
\NormalTok{x <-}\StringTok{ }\DecValTok{1}\OperatorTok{:}\DecValTok{5}
\NormalTok{y <-}\StringTok{ }\KeywordTok{c}\NormalTok{(}\OperatorTok{-}\DecValTok{1}\NormalTok{,}\DecValTok{2}\NormalTok{,}\OperatorTok{-}\DecValTok{3}\NormalTok{,}\DecValTok{4}\NormalTok{,}\OperatorTok{-}\DecValTok{2}\NormalTok{)}
\NormalTok{x[y}\OperatorTok{>}\DecValTok{0}\NormalTok{]}
\end{Highlighting}
\end{Shaded}
\pause
\begin{verbatim}
## [1] 2 4
\end{verbatim}

a menudo necesitamos usar tantas construcciones como sean posibles, las
cuales son llamadas máscaras lógicas, en estos existen dos ventajs: el
código es sencillo para leer y rápido para ejecutar

Note que las funciones which(),which.min(), which.max() serán útiles

\end{frame}

\begin{frame}[fragile]{Extracción e inserción de elementos en vectores}

\begin{Shaded}
\begin{Highlighting}[]
\NormalTok{mask <-}\StringTok{ }\KeywordTok{c}\NormalTok{(}\OtherTok{TRUE}\NormalTok{,}\OtherTok{FALSE}\NormalTok{,}\OtherTok{TRUE}\NormalTok{,}\OtherTok{NA}\NormalTok{,}\OtherTok{FALSE}\NormalTok{,}\OtherTok{FALSE}\NormalTok{,}\OtherTok{TRUE}\NormalTok{)}
\KeywordTok{which}\NormalTok{(mask)}
\end{Highlighting}
\end{Shaded}
\pause
\begin{verbatim}
## [1] 1 3 7
\end{verbatim}

\begin{Shaded}
\begin{Highlighting}[]
\NormalTok{x <-}\StringTok{ }\KeywordTok{c}\NormalTok{(}\DecValTok{0}\OperatorTok{:}\DecValTok{4}\NormalTok{,}\DecValTok{0}\OperatorTok{:}\DecValTok{5}\NormalTok{,}\DecValTok{11}\NormalTok{)}
\KeywordTok{which.min}\NormalTok{(x)}
\end{Highlighting}
\end{Shaded}
\pause
\begin{verbatim}
## [1] 1
\end{verbatim}

\begin{Shaded}
\begin{Highlighting}[]
\KeywordTok{which.max}\NormalTok{(x)}
\end{Highlighting}
\end{Shaded}
\pause
\begin{verbatim}
## [1] 12
\end{verbatim}

\end{frame}

\begin{frame}[fragile]{Remplazamiento en vectores}

Para reemplazar elementos en un vector se hace de una manera similar a
la extracción. Todo lo que necesitamos es seleccionar las elementos como
si usted quisiera extraerlos, y usando \textless{}- se sigue de los
elementos para reemplazar. Por supuesto, usted necesita especificar el
mismo número de elementos entrantes que salientes

\begin{Shaded}
\begin{Highlighting}[]
\NormalTok{z <-}\StringTok{ }\KeywordTok{c}\NormalTok{(}\DecValTok{0}\NormalTok{,}\DecValTok{0}\NormalTok{,}\DecValTok{0}\NormalTok{,}\DecValTok{2}\NormalTok{,}\DecValTok{0}\NormalTok{)}
\NormalTok{z[}\KeywordTok{c}\NormalTok{(}\DecValTok{1}\NormalTok{,}\DecValTok{5}\NormalTok{)] <-}\StringTok{ }\DecValTok{1}
\NormalTok{z}
\end{Highlighting}
\end{Shaded}
\pause
\begin{verbatim}
## [1] 1 0 0 2 1
\end{verbatim}

\begin{Shaded}
\begin{Highlighting}[]
\NormalTok{z[}\KeywordTok{which.max}\NormalTok{(z)] <-}\StringTok{ }\DecValTok{0}
\NormalTok{z}
\end{Highlighting}
\end{Shaded}
\pause
\begin{verbatim}
## [1] 1 0 0 0 1
\end{verbatim}

\begin{Shaded}
\begin{Highlighting}[]
\NormalTok{z[z}\OperatorTok{==}\DecValTok{0}\NormalTok{] <-}\StringTok{ }\DecValTok{8}
\NormalTok{z}
\end{Highlighting}
\end{Shaded}
\pause
\begin{verbatim}
## [1] 1 8 8 8 1
\end{verbatim}

\end{frame}

\begin{frame}[fragile]{Agregar o insertar elementos a un vector
preexistente}

Usamos la función c()

\begin{Shaded}
\begin{Highlighting}[]
\NormalTok{vecA <-}\StringTok{ }\KeywordTok{c}\NormalTok{(}\DecValTok{1}\NormalTok{,}\DecValTok{3}\NormalTok{,}\DecValTok{6}\NormalTok{,}\DecValTok{2}\NormalTok{,}\DecValTok{7}\NormalTok{,}\DecValTok{4}\NormalTok{,}\DecValTok{8}\NormalTok{,}\DecValTok{1}\NormalTok{,}\DecValTok{0}\NormalTok{)}
\NormalTok{vecA}
\end{Highlighting}
\end{Shaded}
\pause
\begin{verbatim}
## [1] 1 3 6 2 7 4 8 1 0
\end{verbatim}

\begin{Shaded}
\begin{Highlighting}[]
\NormalTok{(vecB <-}\StringTok{ }\KeywordTok{c}\NormalTok{(vecA, }\DecValTok{4}\NormalTok{, }\DecValTok{1}\NormalTok{))}
\end{Highlighting}
\end{Shaded}
\pause
\begin{verbatim}
##  [1] 1 3 6 2 7 4 8 1 0 4 1
\end{verbatim}

\begin{Shaded}
\begin{Highlighting}[]
\NormalTok{(vecC <-}\StringTok{ }\KeywordTok{c}\NormalTok{(vecA[}\DecValTok{1}\OperatorTok{:}\DecValTok{4}\NormalTok{], }\DecValTok{8}\NormalTok{, }\DecValTok{5}\NormalTok{, vecA[}\DecValTok{5}\OperatorTok{:}\DecValTok{9}\NormalTok{]))}
\end{Highlighting}
\end{Shaded}
\pause
\begin{verbatim}
##  [1] 1 3 6 2 8 5 7 4 8 1 0
\end{verbatim}

\end{frame}

\begin{frame}[fragile]{Agregar o insertar elementos a un vector
preexistente}

Este mecanismo provee la habilidad para completar un vector cuyo tamaño
no es fijo en el principio

\begin{Shaded}
\begin{Highlighting}[]
\NormalTok{a <-}\StringTok{ }\KeywordTok{c}\NormalTok{()}
\NormalTok{a <-}\StringTok{ }\KeywordTok{c}\NormalTok{(a,}\DecValTok{2}\NormalTok{)}
\NormalTok{a <-}\StringTok{ }\KeywordTok{c}\NormalTok{(a,}\DecValTok{7}\NormalTok{)}
\NormalTok{a}
\end{Highlighting}
\end{Shaded}
\pause
\begin{verbatim}
## [1] 2 7
\end{verbatim}

\end{frame}

\begin{frame}{Ejercicio}

Cree un vecctor altura \textless{}- c(182,150,160,140.5,191) y vector
género genero \textless{}- c(0,1,1,1,0) donde la altura seexpresa en cms
y el género 1 mujer y 0 hombre. extraiga del vector altura, las altura
de los hombres. Use el método de extracción de variables por subíndices,
repitiendo la tarea con una máscara lógica

\end{frame}

\begin{frame}[fragile]{Ejercicio}

Extraiga del siguiente vector todos los números entre 2 y 3

\begin{Shaded}
\begin{Highlighting}[]
\NormalTok{x <-}\StringTok{ }\KeywordTok{c}\NormalTok{(}\FloatTok{0.1}\NormalTok{,}\FloatTok{0.5}\NormalTok{,}\FloatTok{2.1}\NormalTok{,}\FloatTok{3.5}\NormalTok{,}\FloatTok{2.8}\NormalTok{,}\FloatTok{2.7}\NormalTok{,}\FloatTok{1.9}\NormalTok{,}\FloatTok{2.2}\NormalTok{,}\FloatTok{5.6}\NormalTok{)}
\end{Highlighting}
\end{Shaded}

\end{frame}

\begin{frame}{Extracción e inserción en matrices}

-Extracción via X{[}índice fila,índice columna{]}, omitir la primera
componente significa que todas las filas son seleccionada, o en su
debido caso las columnas, cuando las componentes son negativas indican
que elementos no extraer

\begin{itemize}
\tightlist
\item
  Extracción vía máscara lógica X{[}máscara{]}, sabiendo que la matriz
  es de valores lógicos de mismo tamaño que X el cual indica que
  elementos extraer
\end{itemize}

\end{frame}

\begin{frame}[fragile]{Extracción e inserción en matrices}

\begin{Shaded}
\begin{Highlighting}[]
\NormalTok{Mat <-}\StringTok{ }\KeywordTok{matrix}\NormalTok{(}\DecValTok{1}\OperatorTok{:}\DecValTok{12}\NormalTok{,}\DataTypeTok{nrow=}\DecValTok{4}\NormalTok{,}\DataTypeTok{ncol=}\DecValTok{3}\NormalTok{,}\DataTypeTok{byrow=}\OtherTok{TRUE}\NormalTok{)}
\NormalTok{Mat}
\end{Highlighting}
\end{Shaded}
\pause
\begin{verbatim}
##      [,1] [,2] [,3]
## [1,]    1    2    3
## [2,]    4    5    6
## [3,]    7    8    9
## [4,]   10   11   12
\end{verbatim}

\begin{Shaded}
\begin{Highlighting}[]
\NormalTok{Mat[}\DecValTok{2}\NormalTok{,}\DecValTok{3}\NormalTok{]}
\end{Highlighting}
\end{Shaded}
\pause
\begin{verbatim}
## [1] 6
\end{verbatim}

\end{frame}

\begin{frame}[fragile]{Extracción e inserción en matrices}

\begin{Shaded}
\begin{Highlighting}[]
\NormalTok{Mat[,}\DecValTok{1}\NormalTok{]}
\end{Highlighting}
\end{Shaded}
\pause
\begin{verbatim}
## [1]  1  4  7 10
\end{verbatim}

\begin{Shaded}
\begin{Highlighting}[]
\NormalTok{Mat[}\KeywordTok{c}\NormalTok{(}\DecValTok{1}\NormalTok{,}\DecValTok{4}\NormalTok{),]}
\end{Highlighting}
\end{Shaded}
\pause
\begin{verbatim}
##      [,1] [,2] [,3]
## [1,]    1    2    3
## [2,]   10   11   12
\end{verbatim}

\begin{Shaded}
\begin{Highlighting}[]
\NormalTok{Mat[}\DecValTok{3}\NormalTok{,}\OperatorTok{-}\KeywordTok{c}\NormalTok{(}\DecValTok{1}\NormalTok{,}\DecValTok{3}\NormalTok{)]}
\end{Highlighting}
\end{Shaded}
\pause
\begin{verbatim}
## [1] 8
\end{verbatim}

\end{frame}

\begin{frame}[fragile]{Extracción e inserción en matrices}

\begin{Shaded}
\begin{Highlighting}[]
\NormalTok{MatLogical <-}\StringTok{ }\KeywordTok{matrix}\NormalTok{(}\KeywordTok{c}\NormalTok{(}\OtherTok{TRUE}\NormalTok{,}\OtherTok{FALSE}\NormalTok{),}\DataTypeTok{nrow=}\DecValTok{4}\NormalTok{,}\DataTypeTok{ncol=}\DecValTok{3}\NormalTok{)}
\NormalTok{MatLogical}
\end{Highlighting}
\end{Shaded}
\pause
\begin{verbatim}
##       [,1]  [,2]  [,3]
## [1,]  TRUE  TRUE  TRUE
## [2,] FALSE FALSE FALSE
## [3,]  TRUE  TRUE  TRUE
## [4,] FALSE FALSE FALSE
\end{verbatim}

\begin{Shaded}
\begin{Highlighting}[]
\NormalTok{Mat[MatLogical]}
\end{Highlighting}
\end{Shaded}
\pause
\begin{verbatim}
## [1] 1 7 2 8 3 9
\end{verbatim}

\end{frame}

\begin{frame}[fragile]{Extracción e inserción en matrices}

\begin{Shaded}
\begin{Highlighting}[]
\NormalTok{ind <-}\StringTok{ }\KeywordTok{c}\NormalTok{(}\DecValTok{2}\NormalTok{,}\DecValTok{4}\NormalTok{,}\DecValTok{6}\NormalTok{,}\DecValTok{8}\NormalTok{,}\DecValTok{3}\NormalTok{)}
\NormalTok{Mat[ind]}
\end{Highlighting}
\end{Shaded}
\pause
\begin{verbatim}
## [1]  4 10  5 11  7
\end{verbatim}

\end{frame}

\begin{frame}[fragile]{Extracción e inserción en matrices}

Algunas veces la función de extracción cambia la estrucutura de
manipulación

\begin{Shaded}
\begin{Highlighting}[]
\NormalTok{m <-}\StringTok{ }\KeywordTok{matrix}\NormalTok{(}\DecValTok{1}\OperatorTok{:}\DecValTok{6}\NormalTok{,}\DataTypeTok{nrow=}\DecValTok{2}\NormalTok{) ; m}
\end{Highlighting}
\end{Shaded}
\pause
\begin{verbatim}
##      [,1] [,2] [,3]
## [1,]    1    3    5
## [2,]    2    4    6
\end{verbatim}

\begin{Shaded}
\begin{Highlighting}[]
\NormalTok{m[,}\DecValTok{1}\NormalTok{]}
\end{Highlighting}
\end{Shaded}
\pause
\begin{verbatim}
## [1] 1 2
\end{verbatim}

\end{frame}

\begin{frame}[fragile]{Extracción e inserción en matrices}

Pero resulta que yo quería el vector como columna, este problema se
puede arreglar

\begin{Shaded}
\begin{Highlighting}[]
\NormalTok{m[,}\DecValTok{1}\NormalTok{,drop=}\OtherTok{FALSE}\NormalTok{]}
\end{Highlighting}
\end{Shaded}
\pause
\begin{verbatim}
##      [,1]
## [1,]    1
## [2,]    2
\end{verbatim}

\end{frame}

\begin{frame}[fragile]{Extracción e inserción en matrices}

Usando la función which yo puedo alternar subíndices de los elementos de
una matriz los cuales son verificados con la condición

\begin{Shaded}
\begin{Highlighting}[]
\NormalTok{m <-}\StringTok{ }\KeywordTok{matrix}\NormalTok{(}\KeywordTok{c}\NormalTok{(}\DecValTok{1}\NormalTok{,}\DecValTok{2}\NormalTok{,}\DecValTok{3}\NormalTok{,}\DecValTok{1}\NormalTok{,}\DecValTok{2}\NormalTok{,}\DecValTok{3}\NormalTok{,}\DecValTok{2}\NormalTok{,}\DecValTok{1}\NormalTok{,}\DecValTok{3}\NormalTok{),}\DecValTok{3}\NormalTok{,}\DecValTok{3}\NormalTok{)}
\NormalTok{m}
\end{Highlighting}
\end{Shaded}
\pause
\begin{verbatim}
##      [,1] [,2] [,3]
## [1,]    1    1    2
## [2,]    2    2    1
## [3,]    3    3    3
\end{verbatim}

\begin{Shaded}
\begin{Highlighting}[]
\KeywordTok{which}\NormalTok{(m }\OperatorTok{==}\StringTok{ }\DecValTok{1}\NormalTok{)}
\end{Highlighting}
\end{Shaded}
\pause
\begin{verbatim}
## [1] 1 4 8
\end{verbatim}

\end{frame}

\begin{frame}[fragile]{Extracción e inserción en matrices}

para poner los índices como parejas

\begin{Shaded}
\begin{Highlighting}[]
\KeywordTok{which}\NormalTok{(m }\OperatorTok{==}\StringTok{ }\DecValTok{1}\NormalTok{,}\DataTypeTok{arr.ind=}\OtherTok{TRUE}\NormalTok{)}
\end{Highlighting}
\end{Shaded}
\pause
\begin{verbatim}
##      row col
## [1,]   1   1
## [2,]   1   2
## [3,]   2   3
\end{verbatim}

\end{frame}

\begin{frame}[fragile]{Extracción e inserción en matrices}

Para realizarr inserción de elementos en una matriz

\begin{Shaded}
\begin{Highlighting}[]
\NormalTok{m[m}\OperatorTok{!=}\DecValTok{2}\NormalTok{] <-}\StringTok{ }\DecValTok{0}
\NormalTok{m}
\end{Highlighting}
\end{Shaded}
\pause
\begin{verbatim}
##      [,1] [,2] [,3]
## [1,]    0    0    2
## [2,]    2    2    0
## [3,]    0    0    0
\end{verbatim}

\begin{Shaded}
\begin{Highlighting}[]
\NormalTok{Mat <-}\StringTok{ }\NormalTok{Mat[}\OperatorTok{-}\DecValTok{4}\NormalTok{,] ; Mat}
\end{Highlighting}
\end{Shaded}
\pause
\begin{verbatim}
##      [,1] [,2] [,3]
## [1,]    1    2    3
## [2,]    4    5    6
## [3,]    7    8    9
\end{verbatim}

\end{frame}

\begin{frame}[fragile]{Extracción e inserción en matrices}

\begin{Shaded}
\begin{Highlighting}[]
\NormalTok{m[Mat}\OperatorTok{>}\DecValTok{7}\NormalTok{] <-}\StringTok{ }\NormalTok{Mat[Mat}\OperatorTok{>}\DecValTok{7}\NormalTok{]}
\NormalTok{m}
\end{Highlighting}
\end{Shaded}
\pause
\begin{verbatim}
##      [,1] [,2] [,3]
## [1,]    0    0    2
## [2,]    2    2    0
## [3,]    0    8    9
\end{verbatim}

\end{frame}

\begin{frame}[fragile]{Ejercicio}

\begin{Shaded}
\begin{Highlighting}[]
\NormalTok{m1 <-}\StringTok{ }\KeywordTok{matrix}\NormalTok{(}\KeywordTok{c}\NormalTok{(}\DecValTok{0}\NormalTok{,}\DecValTok{22}\NormalTok{,}\DecValTok{0}\NormalTok{,}\DecValTok{23}\NormalTok{,}\DecValTok{34}\NormalTok{,}\DecValTok{0}\NormalTok{,}\DecValTok{0}\NormalTok{,}\DecValTok{0}\NormalTok{,}\DecValTok{28}\NormalTok{),}\DataTypeTok{ncol=}\DecValTok{3}\NormalTok{)}
\NormalTok{m2 <-}\StringTok{ }\KeywordTok{matrix}\NormalTok{(}\KeywordTok{c}\NormalTok{(}\DecValTok{10}\NormalTok{,}\DecValTok{1}\NormalTok{,}\DecValTok{4}\NormalTok{,}\DecValTok{10}\NormalTok{,}\DecValTok{9}\NormalTok{,}\DecValTok{9}\NormalTok{,}\DecValTok{2}\NormalTok{,}\DecValTok{6}\NormalTok{,}\DecValTok{4}\NormalTok{),}\DataTypeTok{ncol=}\DecValTok{3}\NormalTok{)}
\end{Highlighting}
\end{Shaded}

reemplace todos los valores distintos de cero de m1 con el
correspondiente valor en m2,despues remueva la segunda columna de m1

\end{frame}

\begin{frame}[fragile]{Extracción e inserción en listas}

\begin{Shaded}
\begin{Highlighting}[]
\NormalTok{L <-}\StringTok{ }\KeywordTok{list}\NormalTok{(}\DecValTok{12}\NormalTok{,}\KeywordTok{c}\NormalTok{(}\DecValTok{34}\NormalTok{,}\DecValTok{67}\NormalTok{,}\DecValTok{8}\NormalTok{),Mat,}\DecValTok{1}\OperatorTok{:}\DecValTok{15}\NormalTok{,}\KeywordTok{list}\NormalTok{(}\DecValTok{10}\NormalTok{,}\DecValTok{11}\NormalTok{))}
\NormalTok{L}
\end{Highlighting}
\end{Shaded}
\pause
\begin{verbatim}
## [[1]]
## [1] 12
## 
## [[2]]
## [1] 34 67  8
## 
## [[3]]
##      [,1] [,2] [,3]
## [1,]    1    2    3
## [2,]    4    5    6
## [3,]    7    8    9

\end{verbatim}

\end{frame}

\begin{frame}[fragile]{Extracción e inserción en listas}
\pause
\begin{verbatim}

## [[4]]
##  [1]  1  2  3  4  5  6  7  8  9 10 11 12 13 14 15
## 
## [[5]]
## [[5]][[1]]
## [1] 10
## 
## [[5]][[2]]
## [1] 11
\end{verbatim}

\end{frame}

\begin{frame}[fragile]{Extracción e inserción en listas}

\begin{Shaded}
\begin{Highlighting}[]
\NormalTok{L[}\DecValTok{2}\NormalTok{]}
\end{Highlighting}
\end{Shaded}
\pause
\begin{verbatim}
## [[1]]
## [1] 34 67  8
\end{verbatim}

\begin{Shaded}
\begin{Highlighting}[]
\KeywordTok{class}\NormalTok{(L[}\DecValTok{2}\NormalTok{])}
\end{Highlighting}
\end{Shaded}
\pause
\begin{verbatim}
## [1] "list"
\end{verbatim}

\end{frame}

\begin{frame}[fragile]{Extracción e inserción en listas}

\begin{Shaded}
\begin{Highlighting}[]
\NormalTok{L[}\KeywordTok{c}\NormalTok{(}\DecValTok{3}\NormalTok{,}\DecValTok{4}\NormalTok{)]}
\end{Highlighting}
\end{Shaded}
\pause
\begin{verbatim}
## [[1]]
##      [,1] [,2] [,3]
## [1,]    1    2    3
## [2,]    4    5    6
## [3,]    7    8    9
## 
## [[2]]
##  [1]  1  2  3  4  5  6  7  8  9 10 11 12 13 14 15
\end{verbatim}

\end{frame}

\begin{frame}[fragile]{Extracción e inserción en listas}

Para acceder a los elementos de una lista usamos

\begin{Shaded}
\begin{Highlighting}[]
\NormalTok{L[[}\DecValTok{2}\NormalTok{]]}
\end{Highlighting}
\end{Shaded}
\pause
\begin{verbatim}
## [1] 34 67  8
\end{verbatim}

\begin{Shaded}
\begin{Highlighting}[]
\NormalTok{L[[}\DecValTok{2}\NormalTok{]][}\DecValTok{1}\NormalTok{]}
\end{Highlighting}
\end{Shaded}
\pause
\begin{verbatim}
## [1] 34
\end{verbatim}

\begin{Shaded}
\begin{Highlighting}[]
\NormalTok{L[[}\DecValTok{5}\NormalTok{]][[}\DecValTok{2}\NormalTok{]]}
\end{Highlighting}
\end{Shaded}
\pause
\begin{verbatim}
## [1] 11
\end{verbatim}

\end{frame}

\begin{frame}[fragile]{Extracción e inserción en listas}

Para usar indexación recursiva usamos

\begin{Shaded}
\begin{Highlighting}[]
\NormalTok{L[[}\KeywordTok{c}\NormalTok{(}\DecValTok{2}\NormalTok{,}\DecValTok{3}\NormalTok{)]]}
\end{Highlighting}
\end{Shaded}
\pause
\begin{verbatim}
## [1] 8
\end{verbatim}

\end{frame}

\begin{frame}[fragile]{Extracción e inserción en listas}

Para accerder a los elementos de una lista con nombres en cada elemento
usamos

\begin{Shaded}
\begin{Highlighting}[]
\NormalTok{L <-}\StringTok{ }\KeywordTok{list}\NormalTok{(}\DataTypeTok{cars=}\KeywordTok{c}\NormalTok{(}\StringTok{"FORD"}\NormalTok{,}\StringTok{"PEUGEOT"}\NormalTok{),}\DataTypeTok{climate=}
            \KeywordTok{c}\NormalTok{(}\StringTok{"Tropical"}\NormalTok{,}\StringTok{"Temperate"}\NormalTok{))}
\NormalTok{L}
\end{Highlighting}
\end{Shaded}
\pause
\begin{verbatim}
## $cars
## [1] "FORD"    "PEUGEOT"
## 
## $climate
## [1] "Tropical"  "Temperate"
\end{verbatim}

\end{frame}

\begin{frame}[fragile]{Extracción e inserción en listas}

\begin{Shaded}
\begin{Highlighting}[]
\NormalTok{L[[}\StringTok{"cars"}\NormalTok{]][}\DecValTok{2}\NormalTok{]}
\end{Highlighting}
\end{Shaded}
\pause
\begin{verbatim}
## [1] "PEUGEOT"
\end{verbatim}

\begin{Shaded}
\begin{Highlighting}[]
\NormalTok{L}\OperatorTok{$}\NormalTok{cars}
\end{Highlighting}
\end{Shaded}
\pause
\begin{verbatim}
## [1] "FORD"    "PEUGEOT"
\end{verbatim}

\end{frame}

\begin{frame}[fragile]{Extracción e inserción en listas}

\begin{Shaded}
\begin{Highlighting}[]
\NormalTok{L}\OperatorTok{$}\NormalTok{climate[}\DecValTok{2}\NormalTok{] <-}\StringTok{ "Continental"}
\NormalTok{L}
\end{Highlighting}
\end{Shaded}
\pause
\begin{verbatim}
## $cars
## [1] "FORD"    "PEUGEOT"
## 
## $climate
## [1] "Tropical"    "Continental"
\end{verbatim}

\end{frame}

\begin{frame}[fragile]{Extracción e inserción en listas}

El nombre de una columna puede incluir espacios. Para acceder a ella,
usted necesita una marca especial

\begin{Shaded}
\begin{Highlighting}[]
\NormalTok{L <-}\StringTok{ }\KeywordTok{list}\NormalTok{(}\StringTok{"pretty cars"}\NormalTok{=}\KeywordTok{c}\NormalTok{(}\StringTok{"FORD"}\NormalTok{,}\StringTok{"PEUGEOT"}\NormalTok{))}
\NormalTok{L}
\end{Highlighting}
\end{Shaded}
\pause
\begin{verbatim}
## $`pretty cars`
## [1] "FORD"    "PEUGEOT"
\end{verbatim}

\begin{Shaded}
\begin{Highlighting}[]
\NormalTok{L}\OperatorTok{$}\StringTok{"pretty cars"}
\end{Highlighting}
\end{Shaded}
\pause
\begin{verbatim}
## [1] "FORD"    "PEUGEOT"
\end{verbatim}

\end{frame}

\begin{frame}[fragile]{Ejercicios}

\begin{Shaded}
\begin{Highlighting}[]
\NormalTok{peas <-}\StringTok{ }\KeywordTok{c}\NormalTok{(}\DecValTok{4}\NormalTok{,}\DecValTok{7}\NormalTok{,}\DecValTok{6}\NormalTok{,}\DecValTok{5}\NormalTok{,}\DecValTok{6}\NormalTok{,}\DecValTok{7}\NormalTok{)}
\end{Highlighting}
\end{Shaded}

\begin{Shaded}
\begin{Highlighting}[]
\NormalTok{peas[}\DecValTok{4}\NormalTok{]}
\end{Highlighting}
\end{Shaded}
\pause
\begin{verbatim}
## [1] 5
\end{verbatim}

\begin{Shaded}
\begin{Highlighting}[]
\NormalTok{pods <-}\StringTok{ }\KeywordTok{c}\NormalTok{(}\DecValTok{2}\NormalTok{,}\DecValTok{3}\NormalTok{,}\DecValTok{6}\NormalTok{)}
\NormalTok{peas[pods]}
\end{Highlighting}
\end{Shaded}
\pause
\begin{verbatim}
## [1] 7 6 7
\end{verbatim}

\end{frame}

\begin{frame}[fragile]{Ejercicios}

\begin{Shaded}
\begin{Highlighting}[]
\NormalTok{peas[}\KeywordTok{c}\NormalTok{(}\DecValTok{2}\NormalTok{,}\DecValTok{3}\NormalTok{,}\DecValTok{6}\NormalTok{)]}
\end{Highlighting}
\end{Shaded}
\pause
\begin{verbatim}
## [1] 7 6 7
\end{verbatim}

\begin{Shaded}
\begin{Highlighting}[]
\NormalTok{peas[}\OperatorTok{-}\DecValTok{1}\NormalTok{]}
\end{Highlighting}
\end{Shaded}
\pause
\begin{verbatim}
## [1] 7 6 5 6 7
\end{verbatim}

\begin{Shaded}
\begin{Highlighting}[]
\NormalTok{peas[}\OperatorTok{-}\KeywordTok{length}\NormalTok{(peas)]}
\end{Highlighting}
\end{Shaded}
\pause
\begin{verbatim}
## [1] 4 7 6 5 6
\end{verbatim}

\end{frame}

\begin{frame}[fragile]{Ejercicios}

\begin{Shaded}
\begin{Highlighting}[]
\NormalTok{peas[}\DecValTok{1}\OperatorTok{:}\DecValTok{3}\NormalTok{]}
\end{Highlighting}
\end{Shaded}
\pause
\begin{verbatim}
## [1] 4 7 6
\end{verbatim}

\begin{Shaded}
\begin{Highlighting}[]
\NormalTok{peas[}\KeywordTok{seq}\NormalTok{(}\DecValTok{2}\NormalTok{,}\KeywordTok{length}\NormalTok{(peas),}\DecValTok{2}\NormalTok{)]}
\end{Highlighting}
\end{Shaded}
\pause
\begin{verbatim}
## [1] 7 5 7
\end{verbatim}

\begin{Shaded}
\begin{Highlighting}[]
\NormalTok{peas[}\DecValTok{1}\OperatorTok{:}\KeywordTok{length}\NormalTok{(peas) }\OperatorTok\StringTok{ }\DecValTok{2} \OperatorTok{==}\StringTok{ }\DecValTok{0}\NormalTok{]}
\end{Highlighting}
\end{Shaded}
\pause
\begin{verbatim}
## [1] 7 5 7
\end{verbatim}

\end{frame}

\begin{frame}[fragile]{Ejercicios}

\begin{Shaded}
\begin{Highlighting}[]
\NormalTok{y <-}\StringTok{ }\FloatTok{4.3}
\NormalTok{z <-}\StringTok{ }\NormalTok{y[}\OperatorTok{-}\DecValTok{1}\NormalTok{]}
\KeywordTok{length}\NormalTok{(z)}
\end{Highlighting}
\end{Shaded}
\pause
\begin{verbatim}
## [1] 0
\end{verbatim}

\begin{Shaded}
\begin{Highlighting}[]
\NormalTok{y <-}\StringTok{ }\KeywordTok{c}\NormalTok{(}\DecValTok{8}\NormalTok{,}\DecValTok{3}\NormalTok{,}\DecValTok{5}\NormalTok{,}\DecValTok{7}\NormalTok{,}\DecValTok{6}\NormalTok{,}\DecValTok{6}\NormalTok{,}\DecValTok{8}\NormalTok{,}\DecValTok{9}\NormalTok{,}\DecValTok{2}\NormalTok{,}\DecValTok{3}\NormalTok{,}\DecValTok{9}\NormalTok{,}\DecValTok{4}\NormalTok{,}\DecValTok{10}\NormalTok{,}\DecValTok{4}\NormalTok{,}\DecValTok{11}\NormalTok{)}
\end{Highlighting}
\end{Shaded}

\begin{Shaded}
\begin{Highlighting}[]
\KeywordTok{rev}\NormalTok{(}\KeywordTok{sort}\NormalTok{(y))[}\DecValTok{1}\OperatorTok{:}\DecValTok{3}\NormalTok{]}
\end{Highlighting}
\end{Shaded}
\pause
\begin{verbatim}
## [1] 11 10  9
\end{verbatim}

\end{frame}

\begin{frame}[fragile]{Ejercicios}

\begin{Shaded}
\begin{Highlighting}[]
\KeywordTok{sum}\NormalTok{(}\KeywordTok{rev}\NormalTok{(}\KeywordTok{sort}\NormalTok{(y)))[}\DecValTok{1}\OperatorTok{:}\DecValTok{3}\NormalTok{]}
\end{Highlighting}
\end{Shaded}
\pause
\begin{verbatim}
## [1] 95 NA NA
\end{verbatim}

\begin{Shaded}
\begin{Highlighting}[]
\NormalTok{names <-}\StringTok{ }\KeywordTok{c}\NormalTok{(}\StringTok{"Williams"}\NormalTok{,}\StringTok{"Jones"}\NormalTok{,}\StringTok{"Smith"}\NormalTok{,}\StringTok{"Williams"}\NormalTok{,}
           \StringTok{"Jones"}\NormalTok{,}\StringTok{"Williams"}\NormalTok{)}
\end{Highlighting}
\end{Shaded}

\begin{Shaded}
\begin{Highlighting}[]
\KeywordTok{table}\NormalTok{(names)}
\end{Highlighting}
\end{Shaded}
\pause
\begin{verbatim}
## names
##    Jones    Smith Williams 
##        2        1        3
\end{verbatim}

\end{frame}

\begin{frame}[fragile]{Ejercicios}

\begin{Shaded}
\begin{Highlighting}[]
\KeywordTok{unique}\NormalTok{(names)}
\end{Highlighting}
\end{Shaded}
\pause
\begin{verbatim}
## [1] "Williams" "Jones"    "Smith"
\end{verbatim}

\begin{Shaded}
\begin{Highlighting}[]
\KeywordTok{duplicated}\NormalTok{(names)}
\end{Highlighting}
\end{Shaded}
\pause
\begin{verbatim}
## [1] FALSE FALSE FALSE  TRUE  TRUE  TRUE
\end{verbatim}

\begin{Shaded}
\begin{Highlighting}[]
\NormalTok{names[}\OperatorTok{!}\KeywordTok{duplicated}\NormalTok{(names)]}
\end{Highlighting}
\end{Shaded}
\pause
\begin{verbatim}
## [1] "Williams" "Jones"    "Smith"
\end{verbatim}

\end{frame}

\begin{frame}[fragile]{Ejercicios}

\begin{Shaded}
\begin{Highlighting}[]
\NormalTok{salary <-}\StringTok{ }\KeywordTok{c}\NormalTok{(}\DecValTok{42}\NormalTok{,}\DecValTok{42}\NormalTok{,}\DecValTok{48}\NormalTok{,}\DecValTok{42}\NormalTok{,}\DecValTok{42}\NormalTok{,}\DecValTok{42}\NormalTok{)}
\NormalTok{salary[}\OperatorTok{!}\KeywordTok{duplicated}\NormalTok{(names)]}
\end{Highlighting}
\end{Shaded}
\pause
\begin{verbatim}
## [1] 42 42 48
\end{verbatim}

\end{frame}

\begin{frame}[fragile]{Ejercicios}

\begin{Shaded}
\begin{Highlighting}[]
\KeywordTok{mean}\NormalTok{(salary[}\OperatorTok{!}\KeywordTok{duplicated}\NormalTok{(names)])}
\end{Highlighting}
\end{Shaded}
\pause
\begin{verbatim}
## [1] 44
\end{verbatim}

\begin{Shaded}
\begin{Highlighting}[]
\KeywordTok{mean}\NormalTok{(salary[}\OperatorTok{!}\KeywordTok{duplicated}\NormalTok{(salary)])}
\end{Highlighting}
\end{Shaded}
\pause
\begin{verbatim}
## [1] 45
\end{verbatim}

\end{frame}

\end{document}
